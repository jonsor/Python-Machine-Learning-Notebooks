
% Default to the notebook output style

    


% Inherit from the specified cell style.




    
\documentclass[11pt]{article}

    
    
    \usepackage[T1]{fontenc}
    % Nicer default font (+ math font) than Computer Modern for most use cases
    \usepackage{mathpazo}

    % Basic figure setup, for now with no caption control since it's done
    % automatically by Pandoc (which extracts ![](path) syntax from Markdown).
    \usepackage{graphicx}
    % We will generate all images so they have a width \maxwidth. This means
    % that they will get their normal width if they fit onto the page, but
    % are scaled down if they would overflow the margins.
    \makeatletter
    \def\maxwidth{\ifdim\Gin@nat@width>\linewidth\linewidth
    \else\Gin@nat@width\fi}
    \makeatother
    \let\Oldincludegraphics\includegraphics
    % Set max figure width to be 80% of text width, for now hardcoded.
    \renewcommand{\includegraphics}[1]{\Oldincludegraphics[width=.8\maxwidth]{#1}}
    % Ensure that by default, figures have no caption (until we provide a
    % proper Figure object with a Caption API and a way to capture that
    % in the conversion process - todo).
    \usepackage{caption}
    \DeclareCaptionLabelFormat{nolabel}{}
    \captionsetup{labelformat=nolabel}

    \usepackage{adjustbox} % Used to constrain images to a maximum size 
    \usepackage{xcolor} % Allow colors to be defined
    \usepackage{enumerate} % Needed for markdown enumerations to work
    \usepackage{geometry} % Used to adjust the document margins
    \usepackage{amsmath} % Equations
    \usepackage{amssymb} % Equations
    \usepackage{textcomp} % defines textquotesingle
    % Hack from http://tex.stackexchange.com/a/47451/13684:
    \AtBeginDocument{%
        \def\PYZsq{\textquotesingle}% Upright quotes in Pygmentized code
    }
    \usepackage{upquote} % Upright quotes for verbatim code
    \usepackage{eurosym} % defines \euro
    \usepackage[mathletters]{ucs} % Extended unicode (utf-8) support
    \usepackage[utf8x]{inputenc} % Allow utf-8 characters in the tex document
    \usepackage{fancyvrb} % verbatim replacement that allows latex
    \usepackage{grffile} % extends the file name processing of package graphics 
                         % to support a larger range 
    % The hyperref package gives us a pdf with properly built
    % internal navigation ('pdf bookmarks' for the table of contents,
    % internal cross-reference links, web links for URLs, etc.)
    \usepackage{hyperref}
    \usepackage{longtable} % longtable support required by pandoc >1.10
    \usepackage{booktabs}  % table support for pandoc > 1.12.2
    \usepackage[inline]{enumitem} % IRkernel/repr support (it uses the enumerate* environment)
    \usepackage[normalem]{ulem} % ulem is needed to support strikethroughs (\sout)
                                % normalem makes italics be italics, not underlines
    

    
    
    % Colors for the hyperref package
    \definecolor{urlcolor}{rgb}{0,.145,.698}
    \definecolor{linkcolor}{rgb}{.71,0.21,0.01}
    \definecolor{citecolor}{rgb}{.12,.54,.11}

    % ANSI colors
    \definecolor{ansi-black}{HTML}{3E424D}
    \definecolor{ansi-black-intense}{HTML}{282C36}
    \definecolor{ansi-red}{HTML}{E75C58}
    \definecolor{ansi-red-intense}{HTML}{B22B31}
    \definecolor{ansi-green}{HTML}{00A250}
    \definecolor{ansi-green-intense}{HTML}{007427}
    \definecolor{ansi-yellow}{HTML}{DDB62B}
    \definecolor{ansi-yellow-intense}{HTML}{B27D12}
    \definecolor{ansi-blue}{HTML}{208FFB}
    \definecolor{ansi-blue-intense}{HTML}{0065CA}
    \definecolor{ansi-magenta}{HTML}{D160C4}
    \definecolor{ansi-magenta-intense}{HTML}{A03196}
    \definecolor{ansi-cyan}{HTML}{60C6C8}
    \definecolor{ansi-cyan-intense}{HTML}{258F8F}
    \definecolor{ansi-white}{HTML}{C5C1B4}
    \definecolor{ansi-white-intense}{HTML}{A1A6B2}

    % commands and environments needed by pandoc snippets
    % extracted from the output of `pandoc -s`
    \providecommand{\tightlist}{%
      \setlength{\itemsep}{0pt}\setlength{\parskip}{0pt}}
    \DefineVerbatimEnvironment{Highlighting}{Verbatim}{commandchars=\\\{\}}
    % Add ',fontsize=\small' for more characters per line
    \newenvironment{Shaded}{}{}
    \newcommand{\KeywordTok}[1]{\textcolor[rgb]{0.00,0.44,0.13}{\textbf{{#1}}}}
    \newcommand{\DataTypeTok}[1]{\textcolor[rgb]{0.56,0.13,0.00}{{#1}}}
    \newcommand{\DecValTok}[1]{\textcolor[rgb]{0.25,0.63,0.44}{{#1}}}
    \newcommand{\BaseNTok}[1]{\textcolor[rgb]{0.25,0.63,0.44}{{#1}}}
    \newcommand{\FloatTok}[1]{\textcolor[rgb]{0.25,0.63,0.44}{{#1}}}
    \newcommand{\CharTok}[1]{\textcolor[rgb]{0.25,0.44,0.63}{{#1}}}
    \newcommand{\StringTok}[1]{\textcolor[rgb]{0.25,0.44,0.63}{{#1}}}
    \newcommand{\CommentTok}[1]{\textcolor[rgb]{0.38,0.63,0.69}{\textit{{#1}}}}
    \newcommand{\OtherTok}[1]{\textcolor[rgb]{0.00,0.44,0.13}{{#1}}}
    \newcommand{\AlertTok}[1]{\textcolor[rgb]{1.00,0.00,0.00}{\textbf{{#1}}}}
    \newcommand{\FunctionTok}[1]{\textcolor[rgb]{0.02,0.16,0.49}{{#1}}}
    \newcommand{\RegionMarkerTok}[1]{{#1}}
    \newcommand{\ErrorTok}[1]{\textcolor[rgb]{1.00,0.00,0.00}{\textbf{{#1}}}}
    \newcommand{\NormalTok}[1]{{#1}}
    
    % Additional commands for more recent versions of Pandoc
    \newcommand{\ConstantTok}[1]{\textcolor[rgb]{0.53,0.00,0.00}{{#1}}}
    \newcommand{\SpecialCharTok}[1]{\textcolor[rgb]{0.25,0.44,0.63}{{#1}}}
    \newcommand{\VerbatimStringTok}[1]{\textcolor[rgb]{0.25,0.44,0.63}{{#1}}}
    \newcommand{\SpecialStringTok}[1]{\textcolor[rgb]{0.73,0.40,0.53}{{#1}}}
    \newcommand{\ImportTok}[1]{{#1}}
    \newcommand{\DocumentationTok}[1]{\textcolor[rgb]{0.73,0.13,0.13}{\textit{{#1}}}}
    \newcommand{\AnnotationTok}[1]{\textcolor[rgb]{0.38,0.63,0.69}{\textbf{\textit{{#1}}}}}
    \newcommand{\CommentVarTok}[1]{\textcolor[rgb]{0.38,0.63,0.69}{\textbf{\textit{{#1}}}}}
    \newcommand{\VariableTok}[1]{\textcolor[rgb]{0.10,0.09,0.49}{{#1}}}
    \newcommand{\ControlFlowTok}[1]{\textcolor[rgb]{0.00,0.44,0.13}{\textbf{{#1}}}}
    \newcommand{\OperatorTok}[1]{\textcolor[rgb]{0.40,0.40,0.40}{{#1}}}
    \newcommand{\BuiltInTok}[1]{{#1}}
    \newcommand{\ExtensionTok}[1]{{#1}}
    \newcommand{\PreprocessorTok}[1]{\textcolor[rgb]{0.74,0.48,0.00}{{#1}}}
    \newcommand{\AttributeTok}[1]{\textcolor[rgb]{0.49,0.56,0.16}{{#1}}}
    \newcommand{\InformationTok}[1]{\textcolor[rgb]{0.38,0.63,0.69}{\textbf{\textit{{#1}}}}}
    \newcommand{\WarningTok}[1]{\textcolor[rgb]{0.38,0.63,0.69}{\textbf{\textit{{#1}}}}}
    
    
    % Define a nice break command that doesn't care if a line doesn't already
    % exist.
    \def\br{\hspace*{\fill} \\* }
    % Math Jax compatability definitions
    \def\gt{>}
    \def\lt{<}
    % Document parameters
    \title{Python Basics}
    
    
    

    % Pygments definitions
    
\makeatletter
\def\PY@reset{\let\PY@it=\relax \let\PY@bf=\relax%
    \let\PY@ul=\relax \let\PY@tc=\relax%
    \let\PY@bc=\relax \let\PY@ff=\relax}
\def\PY@tok#1{\csname PY@tok@#1\endcsname}
\def\PY@toks#1+{\ifx\relax#1\empty\else%
    \PY@tok{#1}\expandafter\PY@toks\fi}
\def\PY@do#1{\PY@bc{\PY@tc{\PY@ul{%
    \PY@it{\PY@bf{\PY@ff{#1}}}}}}}
\def\PY#1#2{\PY@reset\PY@toks#1+\relax+\PY@do{#2}}

\expandafter\def\csname PY@tok@w\endcsname{\def\PY@tc##1{\textcolor[rgb]{0.73,0.73,0.73}{##1}}}
\expandafter\def\csname PY@tok@c\endcsname{\let\PY@it=\textit\def\PY@tc##1{\textcolor[rgb]{0.25,0.50,0.50}{##1}}}
\expandafter\def\csname PY@tok@cp\endcsname{\def\PY@tc##1{\textcolor[rgb]{0.74,0.48,0.00}{##1}}}
\expandafter\def\csname PY@tok@k\endcsname{\let\PY@bf=\textbf\def\PY@tc##1{\textcolor[rgb]{0.00,0.50,0.00}{##1}}}
\expandafter\def\csname PY@tok@kp\endcsname{\def\PY@tc##1{\textcolor[rgb]{0.00,0.50,0.00}{##1}}}
\expandafter\def\csname PY@tok@kt\endcsname{\def\PY@tc##1{\textcolor[rgb]{0.69,0.00,0.25}{##1}}}
\expandafter\def\csname PY@tok@o\endcsname{\def\PY@tc##1{\textcolor[rgb]{0.40,0.40,0.40}{##1}}}
\expandafter\def\csname PY@tok@ow\endcsname{\let\PY@bf=\textbf\def\PY@tc##1{\textcolor[rgb]{0.67,0.13,1.00}{##1}}}
\expandafter\def\csname PY@tok@nb\endcsname{\def\PY@tc##1{\textcolor[rgb]{0.00,0.50,0.00}{##1}}}
\expandafter\def\csname PY@tok@nf\endcsname{\def\PY@tc##1{\textcolor[rgb]{0.00,0.00,1.00}{##1}}}
\expandafter\def\csname PY@tok@nc\endcsname{\let\PY@bf=\textbf\def\PY@tc##1{\textcolor[rgb]{0.00,0.00,1.00}{##1}}}
\expandafter\def\csname PY@tok@nn\endcsname{\let\PY@bf=\textbf\def\PY@tc##1{\textcolor[rgb]{0.00,0.00,1.00}{##1}}}
\expandafter\def\csname PY@tok@ne\endcsname{\let\PY@bf=\textbf\def\PY@tc##1{\textcolor[rgb]{0.82,0.25,0.23}{##1}}}
\expandafter\def\csname PY@tok@nv\endcsname{\def\PY@tc##1{\textcolor[rgb]{0.10,0.09,0.49}{##1}}}
\expandafter\def\csname PY@tok@no\endcsname{\def\PY@tc##1{\textcolor[rgb]{0.53,0.00,0.00}{##1}}}
\expandafter\def\csname PY@tok@nl\endcsname{\def\PY@tc##1{\textcolor[rgb]{0.63,0.63,0.00}{##1}}}
\expandafter\def\csname PY@tok@ni\endcsname{\let\PY@bf=\textbf\def\PY@tc##1{\textcolor[rgb]{0.60,0.60,0.60}{##1}}}
\expandafter\def\csname PY@tok@na\endcsname{\def\PY@tc##1{\textcolor[rgb]{0.49,0.56,0.16}{##1}}}
\expandafter\def\csname PY@tok@nt\endcsname{\let\PY@bf=\textbf\def\PY@tc##1{\textcolor[rgb]{0.00,0.50,0.00}{##1}}}
\expandafter\def\csname PY@tok@nd\endcsname{\def\PY@tc##1{\textcolor[rgb]{0.67,0.13,1.00}{##1}}}
\expandafter\def\csname PY@tok@s\endcsname{\def\PY@tc##1{\textcolor[rgb]{0.73,0.13,0.13}{##1}}}
\expandafter\def\csname PY@tok@sd\endcsname{\let\PY@it=\textit\def\PY@tc##1{\textcolor[rgb]{0.73,0.13,0.13}{##1}}}
\expandafter\def\csname PY@tok@si\endcsname{\let\PY@bf=\textbf\def\PY@tc##1{\textcolor[rgb]{0.73,0.40,0.53}{##1}}}
\expandafter\def\csname PY@tok@se\endcsname{\let\PY@bf=\textbf\def\PY@tc##1{\textcolor[rgb]{0.73,0.40,0.13}{##1}}}
\expandafter\def\csname PY@tok@sr\endcsname{\def\PY@tc##1{\textcolor[rgb]{0.73,0.40,0.53}{##1}}}
\expandafter\def\csname PY@tok@ss\endcsname{\def\PY@tc##1{\textcolor[rgb]{0.10,0.09,0.49}{##1}}}
\expandafter\def\csname PY@tok@sx\endcsname{\def\PY@tc##1{\textcolor[rgb]{0.00,0.50,0.00}{##1}}}
\expandafter\def\csname PY@tok@m\endcsname{\def\PY@tc##1{\textcolor[rgb]{0.40,0.40,0.40}{##1}}}
\expandafter\def\csname PY@tok@gh\endcsname{\let\PY@bf=\textbf\def\PY@tc##1{\textcolor[rgb]{0.00,0.00,0.50}{##1}}}
\expandafter\def\csname PY@tok@gu\endcsname{\let\PY@bf=\textbf\def\PY@tc##1{\textcolor[rgb]{0.50,0.00,0.50}{##1}}}
\expandafter\def\csname PY@tok@gd\endcsname{\def\PY@tc##1{\textcolor[rgb]{0.63,0.00,0.00}{##1}}}
\expandafter\def\csname PY@tok@gi\endcsname{\def\PY@tc##1{\textcolor[rgb]{0.00,0.63,0.00}{##1}}}
\expandafter\def\csname PY@tok@gr\endcsname{\def\PY@tc##1{\textcolor[rgb]{1.00,0.00,0.00}{##1}}}
\expandafter\def\csname PY@tok@ge\endcsname{\let\PY@it=\textit}
\expandafter\def\csname PY@tok@gs\endcsname{\let\PY@bf=\textbf}
\expandafter\def\csname PY@tok@gp\endcsname{\let\PY@bf=\textbf\def\PY@tc##1{\textcolor[rgb]{0.00,0.00,0.50}{##1}}}
\expandafter\def\csname PY@tok@go\endcsname{\def\PY@tc##1{\textcolor[rgb]{0.53,0.53,0.53}{##1}}}
\expandafter\def\csname PY@tok@gt\endcsname{\def\PY@tc##1{\textcolor[rgb]{0.00,0.27,0.87}{##1}}}
\expandafter\def\csname PY@tok@err\endcsname{\def\PY@bc##1{\setlength{\fboxsep}{0pt}\fcolorbox[rgb]{1.00,0.00,0.00}{1,1,1}{\strut ##1}}}
\expandafter\def\csname PY@tok@kc\endcsname{\let\PY@bf=\textbf\def\PY@tc##1{\textcolor[rgb]{0.00,0.50,0.00}{##1}}}
\expandafter\def\csname PY@tok@kd\endcsname{\let\PY@bf=\textbf\def\PY@tc##1{\textcolor[rgb]{0.00,0.50,0.00}{##1}}}
\expandafter\def\csname PY@tok@kn\endcsname{\let\PY@bf=\textbf\def\PY@tc##1{\textcolor[rgb]{0.00,0.50,0.00}{##1}}}
\expandafter\def\csname PY@tok@kr\endcsname{\let\PY@bf=\textbf\def\PY@tc##1{\textcolor[rgb]{0.00,0.50,0.00}{##1}}}
\expandafter\def\csname PY@tok@bp\endcsname{\def\PY@tc##1{\textcolor[rgb]{0.00,0.50,0.00}{##1}}}
\expandafter\def\csname PY@tok@fm\endcsname{\def\PY@tc##1{\textcolor[rgb]{0.00,0.00,1.00}{##1}}}
\expandafter\def\csname PY@tok@vc\endcsname{\def\PY@tc##1{\textcolor[rgb]{0.10,0.09,0.49}{##1}}}
\expandafter\def\csname PY@tok@vg\endcsname{\def\PY@tc##1{\textcolor[rgb]{0.10,0.09,0.49}{##1}}}
\expandafter\def\csname PY@tok@vi\endcsname{\def\PY@tc##1{\textcolor[rgb]{0.10,0.09,0.49}{##1}}}
\expandafter\def\csname PY@tok@vm\endcsname{\def\PY@tc##1{\textcolor[rgb]{0.10,0.09,0.49}{##1}}}
\expandafter\def\csname PY@tok@sa\endcsname{\def\PY@tc##1{\textcolor[rgb]{0.73,0.13,0.13}{##1}}}
\expandafter\def\csname PY@tok@sb\endcsname{\def\PY@tc##1{\textcolor[rgb]{0.73,0.13,0.13}{##1}}}
\expandafter\def\csname PY@tok@sc\endcsname{\def\PY@tc##1{\textcolor[rgb]{0.73,0.13,0.13}{##1}}}
\expandafter\def\csname PY@tok@dl\endcsname{\def\PY@tc##1{\textcolor[rgb]{0.73,0.13,0.13}{##1}}}
\expandafter\def\csname PY@tok@s2\endcsname{\def\PY@tc##1{\textcolor[rgb]{0.73,0.13,0.13}{##1}}}
\expandafter\def\csname PY@tok@sh\endcsname{\def\PY@tc##1{\textcolor[rgb]{0.73,0.13,0.13}{##1}}}
\expandafter\def\csname PY@tok@s1\endcsname{\def\PY@tc##1{\textcolor[rgb]{0.73,0.13,0.13}{##1}}}
\expandafter\def\csname PY@tok@mb\endcsname{\def\PY@tc##1{\textcolor[rgb]{0.40,0.40,0.40}{##1}}}
\expandafter\def\csname PY@tok@mf\endcsname{\def\PY@tc##1{\textcolor[rgb]{0.40,0.40,0.40}{##1}}}
\expandafter\def\csname PY@tok@mh\endcsname{\def\PY@tc##1{\textcolor[rgb]{0.40,0.40,0.40}{##1}}}
\expandafter\def\csname PY@tok@mi\endcsname{\def\PY@tc##1{\textcolor[rgb]{0.40,0.40,0.40}{##1}}}
\expandafter\def\csname PY@tok@il\endcsname{\def\PY@tc##1{\textcolor[rgb]{0.40,0.40,0.40}{##1}}}
\expandafter\def\csname PY@tok@mo\endcsname{\def\PY@tc##1{\textcolor[rgb]{0.40,0.40,0.40}{##1}}}
\expandafter\def\csname PY@tok@ch\endcsname{\let\PY@it=\textit\def\PY@tc##1{\textcolor[rgb]{0.25,0.50,0.50}{##1}}}
\expandafter\def\csname PY@tok@cm\endcsname{\let\PY@it=\textit\def\PY@tc##1{\textcolor[rgb]{0.25,0.50,0.50}{##1}}}
\expandafter\def\csname PY@tok@cpf\endcsname{\let\PY@it=\textit\def\PY@tc##1{\textcolor[rgb]{0.25,0.50,0.50}{##1}}}
\expandafter\def\csname PY@tok@c1\endcsname{\let\PY@it=\textit\def\PY@tc##1{\textcolor[rgb]{0.25,0.50,0.50}{##1}}}
\expandafter\def\csname PY@tok@cs\endcsname{\let\PY@it=\textit\def\PY@tc##1{\textcolor[rgb]{0.25,0.50,0.50}{##1}}}

\def\PYZbs{\char`\\}
\def\PYZus{\char`\_}
\def\PYZob{\char`\{}
\def\PYZcb{\char`\}}
\def\PYZca{\char`\^}
\def\PYZam{\char`\&}
\def\PYZlt{\char`\<}
\def\PYZgt{\char`\>}
\def\PYZsh{\char`\#}
\def\PYZpc{\char`\%}
\def\PYZdl{\char`\$}
\def\PYZhy{\char`\-}
\def\PYZsq{\char`\'}
\def\PYZdq{\char`\"}
\def\PYZti{\char`\~}
% for compatibility with earlier versions
\def\PYZat{@}
\def\PYZlb{[}
\def\PYZrb{]}
\makeatother


    % Exact colors from NB
    \definecolor{incolor}{rgb}{0.0, 0.0, 0.5}
    \definecolor{outcolor}{rgb}{0.545, 0.0, 0.0}



    
    % Prevent overflowing lines due to hard-to-break entities
    \sloppy 
    % Setup hyperref package
    \hypersetup{
      breaklinks=true,  % so long urls are correctly broken across lines
      colorlinks=true,
      urlcolor=urlcolor,
      linkcolor=linkcolor,
      citecolor=citecolor,
      }
    % Slightly bigger margins than the latex defaults
    
    \geometry{verbose,tmargin=1in,bmargin=1in,lmargin=1in,rmargin=1in}
    
    

    \begin{document}
    
    
    \maketitle
    
    

    
    \section{Python Basics}\label{python-basics}

Python is an \textbf{interpreted, object-oriented, high-level
programming language} with dynamic semantics. Its high-level built in
data structures, combined with \textbf{dynamic typing} and
\textbf{dynamic binding}, make it very attractive for Rapid Application
Development, as well as for use as a scripting or glue language to
connect existing components together. Python's simple, \textbf{easy to
learn} syntax emphasizes \textbf{readability} and therefore reduces the
cost of program maintenance. Python supports \textbf{modules and
packages}, which encourages program \textbf{modularity and code reuse}.
The Python interpreter and the extensive standard library are available
in source or binary form without charge for all major platforms, and can
be freely distributed.
\textbf{\href{https://www.python.org/doc/essays/blurb/}{ref}}

    \subsection{Installing and importing
libraries}\label{installing-and-importing-libraries}

A library is a collection of functions and methods that are not included
in the base version of Python. In Python it is (for the most part) very
easy to install and use libraries.

    \begin{Verbatim}[commandchars=\\\{\}]
{\color{incolor}In [{\color{incolor}1}]:} \PY{c+c1}{\PYZsh{}Installing sci\PYZhy{}kit learn and numpy}
        
        \PY{c+c1}{\PYZsh{}Machine learning library}
        \PY{o}{!}pip install scikit\PYZhy{}learn
        
        \PY{c+c1}{\PYZsh{}Math and data structures}
        \PY{o}{!}pip install numpy
        
        \PY{c+c1}{\PYZsh{}Importing libraries}
        \PY{k+kn}{import} \PY{n+nn}{numpy} \PY{k}{as} \PY{n+nn}{np} \PY{c+c1}{\PYZsh{}Import all of numpy with namespace np}
        \PY{k+kn}{from} \PY{n+nn}{sklearn}\PY{n+nn}{.}\PY{n+nn}{linear\PYZus{}model} \PY{k}{import} \PY{n}{LinearRegression} \PY{c+c1}{\PYZsh{}Import linear regression from sklearn}
\end{Verbatim}


    \begin{Verbatim}[commandchars=\\\{\}]
Requirement already satisfied: scikit-learn in c:\textbackslash{}users\textbackslash{}jsors\textbackslash{}anaconda3\textbackslash{}lib\textbackslash{}site-packages (0.19.2)
Requirement already satisfied: numpy in c:\textbackslash{}users\textbackslash{}jsors\textbackslash{}anaconda3\textbackslash{}lib\textbackslash{}site-packages (1.15.4)

    \end{Verbatim}

    \subsection{Syntax}\label{syntax}

Python has no mandatory termintaion characters (;), blocks are specified
by indentation.

Comments start with a hashtag (\#), or are encapsulated by a ''' comment
''' block.

    \subsection{Variables and Assignments}\label{variables-and-assignments}

    \begin{Verbatim}[commandchars=\\\{\}]
{\color{incolor}In [{\color{incolor}2}]:} \PY{n}{a} \PY{o}{=} \PY{l+m+mi}{42} \PY{c+c1}{\PYZsh{} integer}
        \PY{n}{a} \PY{o}{+}\PY{o}{=} \PY{l+m+mi}{1} \PY{c+c1}{\PYZsh{} increment by one}
        
        \PY{n}{b} \PY{o}{=} \PY{l+m+mf}{3.14} \PY{c+c1}{\PYZsh{} float}
        \PY{n}{do} \PY{o}{=} \PY{l+m+mf}{3.1423985892358923895239858923589289358922359233463462309582350982350982350925309823509835892538925398} \PY{c+c1}{\PYZsh{} float}
        \PY{n}{c} \PY{o}{=} \PY{l+s+s2}{\PYZdq{}}\PY{l+s+s2}{Hello}\PY{l+s+s2}{\PYZdq{}} \PY{c+c1}{\PYZsh{} String}
        \PY{n}{d} \PY{o}{=} \PY{k+kc}{True} \PY{c+c1}{\PYZsh{} Boolean}
        
        \PY{n+nb}{print}\PY{p}{(}\PY{n+nb}{type}\PY{p}{(}\PY{n}{c}\PY{p}{)}\PY{p}{)}
\end{Verbatim}


    \begin{Verbatim}[commandchars=\\\{\}]
<class 'str'>

    \end{Verbatim}

    \subsection{Printing}\label{printing}

    \begin{Verbatim}[commandchars=\\\{\}]
{\color{incolor}In [{\color{incolor}3}]:} \PY{n+nb}{print}\PY{p}{(}\PY{l+s+s2}{\PYZdq{}}\PY{l+s+s2}{Hello World!}\PY{l+s+s2}{\PYZdq{}}\PY{p}{)}
        
        \PY{n+nb}{print}\PY{p}{(}\PY{l+m+mi}{10}\PY{p}{)}
        
        \PY{n+nb}{print}\PY{p}{(}\PY{l+s+s2}{\PYZdq{}}\PY{l+s+s2}{number 1: }\PY{l+s+s2}{\PYZdq{}}\PY{p}{,} \PY{l+m+mi}{5}\PY{p}{,} \PY{l+s+s2}{\PYZdq{}}\PY{l+s+s2}{number 2: }\PY{l+s+s2}{\PYZdq{}}\PY{p}{,} \PY{l+m+mi}{25}\PY{p}{)}
        
        \PY{n+nb}{print}\PY{p}{(}\PY{l+s+s2}{\PYZdq{}}\PY{l+s+s2}{The number: }\PY{l+s+s2}{\PYZdq{}} \PY{o}{+} \PY{n+nb}{str}\PY{p}{(}\PY{l+m+mi}{10}\PY{p}{)}\PY{p}{)}
\end{Verbatim}


    \begin{Verbatim}[commandchars=\\\{\}]
Hello World!
10
number 1:  5 number 2:  25
The number: 10

    \end{Verbatim}

    \subsection{Getting help}\label{getting-help}

    \begin{Verbatim}[commandchars=\\\{\}]
{\color{incolor}In [{\color{incolor}4}]:} \PY{c+c1}{\PYZsh{}print(help(5)) \PYZsh{} Explains how an object works}
        
        \PY{c+c1}{\PYZsh{}print(dir(5)) \PYZsh{} Shows all the methods of an object}
        
        \PY{c+c1}{\PYZsh{}print(abs.\PYZus{}\PYZus{}doc\PYZus{}\PYZus{}) \PYZsh{} Shows documentation of an object}
\end{Verbatim}


    \subsection{Basic calculation}\label{basic-calculation}

    \begin{Verbatim}[commandchars=\\\{\}]
{\color{incolor}In [{\color{incolor}5}]:} \PY{n+nb}{print}\PY{p}{(}\PY{l+m+mi}{3}\PY{o}{+}\PY{l+m+mi}{4}\PY{p}{)} \PY{c+c1}{\PYZsh{} Addition}
        \PY{n+nb}{print}\PY{p}{(}\PY{l+m+mf}{1.6} \PY{o}{*} \PY{l+m+mi}{15}\PY{p}{)} \PY{c+c1}{\PYZsh{} Multiplication}
        \PY{n+nb}{print}\PY{p}{(}\PY{l+m+mi}{6}\PY{o}{/}\PY{l+m+mi}{9}\PY{p}{)} \PY{c+c1}{\PYZsh{} Division}
        \PY{n+nb}{print}\PY{p}{(}\PY{l+m+mi}{10}\PY{o}{*} \PY{l+s+s2}{\PYZdq{}}\PY{l+s+s2}{abc}\PY{l+s+s2}{\PYZdq{}}\PY{p}{)} \PY{c+c1}{\PYZsh{} String multiplication}
        \PY{n+nb}{print}\PY{p}{(}\PY{l+m+mi}{2}\PY{o}{*}\PY{o}{*}\PY{l+m+mi}{7}\PY{p}{)} \PY{c+c1}{\PYZsh{} Exponentiation}
        
        \PY{n}{my\PYZus{}string} \PY{o}{=} \PY{l+s+s2}{\PYZdq{}}\PY{l+s+s2}{Hello}\PY{l+s+s2}{\PYZdq{}} \PY{o}{+} \PY{l+s+s1}{\PYZsq{}}\PY{l+s+s1}{ }\PY{l+s+s1}{\PYZsq{}} \PY{o}{+} \PY{l+s+s2}{\PYZdq{}}\PY{l+s+s2}{World!}\PY{l+s+s2}{\PYZdq{}} \PY{c+c1}{\PYZsh{} Concatenating string}
        \PY{n+nb}{print}\PY{p}{(}\PY{n}{my\PYZus{}string}\PY{p}{)}
        \PY{n+nb}{print}\PY{p}{(}\PY{l+m+mi}{500} \PY{o}{==} \PY{l+m+mi}{500}\PY{p}{)} \PY{c+c1}{\PYZsh{} Equals}
        \PY{n+nb}{print}\PY{p}{(}\PY{l+s+s2}{\PYZdq{}}\PY{l+s+s2}{House}\PY{l+s+s2}{\PYZdq{}} \PY{o}{!=} \PY{l+s+s2}{\PYZdq{}}\PY{l+s+s2}{Car}\PY{l+s+s2}{\PYZdq{}}\PY{p}{)} \PY{c+c1}{\PYZsh{} Not equal}
        \PY{n+nb}{print}\PY{p}{(}\PY{l+m+mi}{100} \PY{o}{\PYZgt{}}\PY{o}{=} \PY{l+m+mi}{100}\PY{p}{)} \PY{c+c1}{\PYZsh{} More or equal to}
        \PY{n+nb}{print}\PY{p}{(}\PY{l+m+mi}{5} \PY{o}{\PYZlt{}} \PY{l+m+mi}{1}\PY{p}{)} \PY{c+c1}{\PYZsh{} Less than}
\end{Verbatim}


    \begin{Verbatim}[commandchars=\\\{\}]
7
24.0
0.6666666666666666
abcabcabcabcabcabcabcabcabcabc
128
Hello World!
True
True
True
False

    \end{Verbatim}

    \subsection{Data Structures}\label{data-structures}

The data structures in python are:

\textbf{Lists:} One dimensional arrays

\textbf{Dictionaries:} Associative arrays (Hash tables)

\textbf{Tuples:} Immutable one-dimensional arrays

Python lists can be of any type, and you can mix types such as strings
and integers

    \subsubsection{Lists}\label{lists}

    \begin{Verbatim}[commandchars=\\\{\}]
{\color{incolor}In [{\color{incolor}6}]:} \PY{c+c1}{\PYZsh{} Make a list}
        \PY{n}{integer\PYZus{}list} \PY{o}{=} \PY{p}{[}\PY{l+m+mi}{1}\PY{p}{,} \PY{l+m+mi}{2}\PY{p}{,} \PY{l+m+mi}{3}\PY{p}{,} \PY{l+m+mi}{4}\PY{p}{,} \PY{l+m+mi}{5}\PY{p}{,} \PY{l+m+mi}{6}\PY{p}{,} \PY{l+m+mi}{7}\PY{p}{,} \PY{l+m+mi}{8}\PY{p}{,} \PY{l+m+mi}{9}\PY{p}{,} \PY{l+m+mi}{10}\PY{p}{]}
        
        \PY{n}{number\PYZus{}list} \PY{o}{=} \PY{p}{[}\PY{l+m+mi}{1}\PY{p}{,} \PY{l+m+mi}{2}\PY{p}{,} \PY{l+m+mf}{6.123}\PY{p}{,}\PY{l+m+mi}{6000}\PY{p}{,} \PY{l+m+mf}{9.9999999231}\PY{p}{,} \PY{l+m+mi}{5}\PY{p}{]}
        
        \PY{n}{mixed\PYZus{}list} \PY{o}{=} \PY{p}{[}\PY{l+s+s2}{\PYZdq{}}\PY{l+s+s2}{String}\PY{l+s+s2}{\PYZdq{}}\PY{p}{,} \PY{l+m+mi}{10}\PY{p}{,} \PY{n}{number\PYZus{}list}\PY{p}{,} \PY{p}{(}\PY{l+s+s2}{\PYZdq{}}\PY{l+s+s2}{a}\PY{l+s+s2}{\PYZdq{}}\PY{p}{,} \PY{l+s+s2}{\PYZdq{}}\PY{l+s+s2}{tuple}\PY{l+s+s2}{\PYZdq{}}\PY{p}{)}\PY{p}{]}
        
        \PY{n}{squares\PYZus{}list} \PY{o}{=} \PY{p}{[}\PY{n}{x}\PY{o}{*}\PY{o}{*}\PY{l+m+mi}{2} \PY{k}{for} \PY{n}{x} \PY{o+ow}{in} \PY{n+nb}{range}\PY{p}{(}\PY{l+m+mi}{1}\PY{p}{,} \PY{l+m+mi}{11}\PY{p}{)}\PY{p}{]}
        
        \PY{c+c1}{\PYZsh{} Change item in list}
        \PY{n}{mixed\PYZus{}list}\PY{p}{[}\PY{l+m+mi}{0}\PY{p}{]} \PY{o}{=} \PY{l+s+s2}{\PYZdq{}}\PY{l+s+s2}{New string}\PY{l+s+s2}{\PYZdq{}}
        
        \PY{c+c1}{\PYZsh{}Add item to list}
        \PY{n}{mixed\PYZus{}list}\PY{o}{.}\PY{n}{append}\PY{p}{(}\PY{l+s+s2}{\PYZdq{}}\PY{l+s+s2}{New item}\PY{l+s+s2}{\PYZdq{}}\PY{p}{)}
        
        \PY{n+nb}{print}\PY{p}{(}\PY{l+s+s2}{\PYZdq{}}\PY{l+s+s2}{mixed\PYZus{}list: }\PY{l+s+s2}{\PYZdq{}}\PY{p}{,} \PY{n}{mixed\PYZus{}list}\PY{p}{)}
        \PY{n+nb}{print}\PY{p}{(}\PY{l+s+s2}{\PYZdq{}}\PY{l+s+s2}{squares\PYZus{}list: }\PY{l+s+s2}{\PYZdq{}}\PY{p}{,} \PY{n}{squares\PYZus{}list}\PY{p}{)}
        
        \PY{c+c1}{\PYZsh{} Conditional tests with lists}
        \PY{n+nb}{print}\PY{p}{(}\PY{l+s+s2}{\PYZdq{}}\PY{l+s+s2}{New string}\PY{l+s+s2}{\PYZdq{}} \PY{o+ow}{in} \PY{n}{mixed\PYZus{}list}\PY{p}{)}
        \PY{n+nb}{print}\PY{p}{(}\PY{l+s+s2}{\PYZdq{}}\PY{l+s+s2}{Other string}\PY{l+s+s2}{\PYZdq{}} \PY{o+ow}{not} \PY{o+ow}{in} \PY{n}{mixed\PYZus{}list}\PY{p}{)}
        
        \PY{c+c1}{\PYZsh{} Simple statistics}
        \PY{n}{smallest\PYZus{}number} \PY{o}{=} \PY{n+nb}{min}\PY{p}{(}\PY{n}{number\PYZus{}list}\PY{p}{)}
        \PY{n}{largest\PYZus{}number} \PY{o}{=} \PY{n+nb}{max}\PY{p}{(}\PY{n}{number\PYZus{}list}\PY{p}{)}
        \PY{n}{sum\PYZus{}of\PYZus{}all\PYZus{}numbers\PYZus{}in\PYZus{}list} \PY{o}{=} \PY{n+nb}{sum}\PY{p}{(}\PY{n}{number\PYZus{}list}\PY{p}{)}
        
        \PY{c+c1}{\PYZsh{} Copy list}
        \PY{n}{mixed\PYZus{}list\PYZus{}copy} \PY{o}{=} \PY{n}{number\PYZus{}list}\PY{p}{[}\PY{p}{:}\PY{p}{]}
        \PY{n}{mixed\PYZus{}list\PYZus{}copy}\PY{p}{[}\PY{l+m+mi}{0}\PY{p}{]} \PY{o}{=} \PY{l+m+mi}{500}
        
        \PY{n+nb}{print}\PY{p}{(}\PY{n}{number\PYZus{}list}\PY{p}{)}
        \PY{n+nb}{print}\PY{p}{(}\PY{n}{mixed\PYZus{}list\PYZus{}copy}\PY{p}{)}
\end{Verbatim}


    \begin{Verbatim}[commandchars=\\\{\}]
mixed\_list:  ['New string', 10, [1, 2, 6.123, 6000, 9.9999999231, 5], ('a', 'tuple'), 'New item']
squares\_list:  [1, 4, 9, 16, 25, 36, 49, 64, 81, 100]
True
True
[1, 2, 6.123, 6000, 9.9999999231, 5]
[500, 2, 6.123, 6000, 9.9999999231, 5]

    \end{Verbatim}

    \subsubsection{Tuples and dictionaries}\label{tuples-and-dictionaries}

    \begin{Verbatim}[commandchars=\\\{\}]
{\color{incolor}In [{\color{incolor}7}]:} \PY{c+c1}{\PYZsh{} Create Tuple}
        \PY{n}{my\PYZus{}tuple} \PY{o}{=} \PY{p}{(}\PY{l+s+s2}{\PYZdq{}}\PY{l+s+s2}{a}\PY{l+s+s2}{\PYZdq{}}\PY{p}{,} \PY{l+s+s2}{\PYZdq{}}\PY{l+s+s2}{tuple}\PY{l+s+s2}{\PYZdq{}}\PY{p}{)}
        
        \PY{c+c1}{\PYZsh{} Dictionaries}
        \PY{n}{my\PYZus{}dict} \PY{o}{=} \PY{p}{\PYZob{}}\PY{l+s+s2}{\PYZdq{}}\PY{l+s+s2}{Key 1}\PY{l+s+s2}{\PYZdq{}}\PY{p}{:} \PY{l+s+s2}{\PYZdq{}}\PY{l+s+s2}{Value 1}\PY{l+s+s2}{\PYZdq{}}\PY{p}{,} \PY{l+m+mi}{2}\PY{p}{:} \PY{l+m+mi}{3}\PY{p}{,} \PY{l+s+s2}{\PYZdq{}}\PY{l+s+s2}{pi}\PY{l+s+s2}{\PYZdq{}}\PY{p}{:} \PY{l+m+mf}{3.14}\PY{p}{\PYZcb{}}
        \PY{n}{my\PYZus{}dict}\PY{p}{[}\PY{l+s+s2}{\PYZdq{}}\PY{l+s+s2}{pi}\PY{l+s+s2}{\PYZdq{}}\PY{p}{]} \PY{o}{=} \PY{l+m+mf}{3.18}
        \PY{n+nb}{print}\PY{p}{(}\PY{n}{my\PYZus{}dict}\PY{p}{[}\PY{l+m+mi}{2}\PY{p}{]}\PY{p}{)}
\end{Verbatim}


    \begin{Verbatim}[commandchars=\\\{\}]
3

    \end{Verbatim}

    You can access \textbf{array ranges} using a colon(:). The number before
the colon specifies where the array range begins, and the one after
specifies where it ends. So some\_list{[}3:12{]} will return all the
items between index 3 and 12.

    \begin{Verbatim}[commandchars=\\\{\}]
{\color{incolor}In [{\color{incolor}8}]:} \PY{c+c1}{\PYZsh{} Lists all items}
        \PY{n+nb}{print}\PY{p}{(}\PY{n}{integer\PYZus{}list}\PY{p}{[}\PY{p}{:}\PY{p}{]}\PY{p}{)}
        \PY{c+c1}{\PYZsh{} Lists all items from index 0 to 2}
        \PY{n+nb}{print}\PY{p}{(}\PY{n}{integer\PYZus{}list}\PY{p}{[}\PY{l+m+mi}{3}\PY{p}{:}\PY{l+m+mi}{8}\PY{p}{]}\PY{p}{)}
        
        \PY{c+c1}{\PYZsh{} Negative indexes count from the last item backwards}
        \PY{c+c1}{\PYZsh{}(thus \PYZhy{}1 is the last item)}
        \PY{n+nb}{print}\PY{p}{(}\PY{n}{integer\PYZus{}list}\PY{p}{[}\PY{o}{\PYZhy{}}\PY{l+m+mi}{3}\PY{p}{]}\PY{p}{)}
        \PY{n+nb}{print}\PY{p}{(}\PY{n}{integer\PYZus{}list}\PY{p}{[}\PY{o}{\PYZhy{}}\PY{l+m+mi}{3}\PY{p}{:}\PY{o}{\PYZhy{}}\PY{l+m+mi}{1}\PY{p}{]}\PY{p}{)}
        
        \PY{c+c1}{\PYZsh{} Two colons together can be used to have python step in N increments}
        \PY{n+nb}{print}\PY{p}{(}\PY{n}{integer\PYZus{}list}\PY{p}{[}\PY{p}{:}\PY{p}{:}\PY{l+m+mi}{2}\PY{p}{]}\PY{p}{)}
        \PY{n+nb}{print}\PY{p}{(}\PY{n}{integer\PYZus{}list}\PY{p}{[}\PY{p}{:}\PY{p}{:}\PY{l+m+mi}{3}\PY{p}{]}\PY{p}{)}
        
        \PY{c+c1}{\PYZsh{} Combined}
        \PY{n+nb}{print}\PY{p}{(}\PY{n}{integer\PYZus{}list}\PY{p}{[}\PY{l+m+mi}{1}\PY{p}{:}\PY{l+m+mi}{6}\PY{p}{:}\PY{l+m+mi}{2}\PY{p}{]}\PY{p}{)}
\end{Verbatim}


    \begin{Verbatim}[commandchars=\\\{\}]
[1, 2, 3, 4, 5, 6, 7, 8, 9, 10]
[4, 5, 6, 7, 8]
8
[8, 9]
[1, 3, 5, 7, 9]
[1, 4, 7, 10]
[2, 4, 6]

    \end{Verbatim}

    \section{Statements and loops}\label{statements-and-loops}

    \begin{Verbatim}[commandchars=\\\{\}]
{\color{incolor}In [{\color{incolor}9}]:} \PY{c+c1}{\PYZsh{} If else statement}
        
        \PY{n}{i} \PY{o}{=} \PY{l+m+mi}{4}
        \PY{k}{if} \PY{n}{i} \PY{o}{==} \PY{l+m+mi}{5}\PY{p}{:}
            \PY{n+nb}{print}\PY{p}{(}\PY{l+s+s2}{\PYZdq{}}\PY{l+s+s2}{hei}\PY{l+s+s2}{\PYZdq{}}\PY{p}{)}
        \PY{k}{else}\PY{p}{:}
            \PY{n+nb}{print}\PY{p}{(}\PY{l+s+s2}{\PYZdq{}}\PY{l+s+s2}{nei}\PY{l+s+s2}{\PYZdq{}}\PY{p}{)}
        
        
        \PY{c+c1}{\PYZsh{} For loop in range}
        \PY{k}{for} \PY{n}{x} \PY{o+ow}{in} \PY{n+nb}{range}\PY{p}{(}\PY{l+m+mi}{3}\PY{p}{,} \PY{l+m+mi}{10}\PY{p}{)}\PY{p}{:}
            \PY{n+nb}{print}\PY{p}{(}\PY{n}{x}\PY{p}{)}
        \PY{k}{else}\PY{p}{:}
            \PY{n+nb}{print}\PY{p}{(}\PY{l+s+s2}{\PYZdq{}}\PY{l+s+s2}{end: }\PY{l+s+s2}{\PYZdq{}}\PY{p}{,} \PY{n}{x}\PY{p}{)}
        
        \PY{c+c1}{\PYZsh{} For loop on list}
        \PY{k}{for} \PY{n}{element} \PY{o+ow}{in} \PY{n}{mixed\PYZus{}list}\PY{p}{:}
            \PY{n+nb}{print}\PY{p}{(}\PY{n}{element}\PY{p}{)}
        
        \PY{k}{for} \PY{n}{j} \PY{o+ow}{in} \PY{n+nb}{range}\PY{p}{(}\PY{n+nb}{len}\PY{p}{(}\PY{n}{mixed\PYZus{}list}\PY{p}{)}\PY{p}{)}\PY{p}{:}
            \PY{n+nb}{print}\PY{p}{(}\PY{n}{mixed\PYZus{}list}\PY{p}{[}\PY{n}{j}\PY{p}{]}\PY{p}{)}
            
        \PY{c+c1}{\PYZsh{} Looping dictionary}
        \PY{c+c1}{\PYZsh{} You can also loop keys with dict.keys(), }
        \PY{c+c1}{\PYZsh{}and all values with dict.values}
        \PY{k}{for} \PY{n}{key}\PY{p}{,} \PY{n}{value} \PY{o+ow}{in} \PY{n}{my\PYZus{}dict}\PY{o}{.}\PY{n}{items}\PY{p}{(}\PY{p}{)}\PY{p}{:}
            \PY{n+nb}{print}\PY{p}{(}\PY{l+s+s2}{\PYZdq{}}\PY{l+s+s2}{Key: }\PY{l+s+s2}{\PYZdq{}}\PY{p}{,} \PY{n}{key}\PY{p}{,} \PY{l+s+s2}{\PYZdq{}}\PY{l+s+s2}{ Value: }\PY{l+s+s2}{\PYZdq{}}\PY{p}{,} \PY{n}{value}\PY{p}{)}
            
        \PY{c+c1}{\PYZsh{} While loop}
        \PY{n}{d} \PY{o}{=} \PY{l+m+mi}{0}
        \PY{k}{while} \PY{n}{d} \PY{o}{\PYZlt{}} \PY{l+m+mi}{10}\PY{p}{:}
            \PY{n+nb}{print}\PY{p}{(}\PY{n}{d}\PY{p}{)}
            \PY{n}{d} \PY{o}{+}\PY{o}{=} \PY{l+m+mi}{1}
            
\end{Verbatim}


    \begin{Verbatim}[commandchars=\\\{\}]
nei
3
4
5
6
7
8
9
end:  9
New string
10
[1, 2, 6.123, 6000, 9.9999999231, 5]
('a', 'tuple')
New item
New string
10
[1, 2, 6.123, 6000, 9.9999999231, 5]
('a', 'tuple')
New item
Key:  Key 1  Value:  Value 1
Key:  2  Value:  3
Key:  pi  Value:  3.18
0
1
2
3
4
5
6
7
8
9

    \end{Verbatim}

    \section{Functions}\label{functions}

Functions are declared with the def keyword. Parameters are passed by
reference, and can be given default values. Functions can return
multiple variables as tuples.

    \begin{Verbatim}[commandchars=\\\{\}]
{\color{incolor}In [{\color{incolor}10}]:} \PY{k}{def} \PY{n+nf}{my\PYZus{}func}\PY{p}{(}\PY{n}{var1} \PY{o}{=} \PY{l+m+mi}{10}\PY{p}{,} \PY{n}{some\PYZus{}string} \PY{o}{=} \PY{l+s+s2}{\PYZdq{}}\PY{l+s+s2}{Hello}\PY{l+s+s2}{\PYZdq{}}\PY{p}{)}\PY{p}{:}
             \PY{n}{x} \PY{o}{=} \PY{l+m+mi}{5} \PY{o}{+} \PY{n}{var1}
             \PY{n}{y} \PY{o}{=} \PY{l+s+s2}{\PYZdq{}}\PY{l+s+s2}{hello }\PY{l+s+s2}{\PYZdq{}} \PY{o}{+}  \PY{n}{some\PYZus{}string}
             \PY{n}{z} \PY{o}{=} \PY{l+m+mf}{5.5}
             
             \PY{k}{return} \PY{n}{x}\PY{p}{,} \PY{n}{y}\PY{p}{,} \PY{n}{z}
         
         \PY{n+nb}{print}\PY{p}{(}\PY{n}{my\PYZus{}func}\PY{p}{(}\PY{p}{)}\PY{p}{)}
         
         
         \PY{c+c1}{\PYZsh{} Lambda functions, single statement functions}
         \PY{c+c1}{\PYZsh{} Same as def lambda\PYZus{}func(x): return x+5}
         
         \PY{n}{lambda\PYZus{}func} \PY{o}{=} \PY{k}{lambda} \PY{n}{x}\PY{p}{:} \PY{n}{x}\PY{o}{+}\PY{l+m+mi}{5}
\end{Verbatim}


    \begin{Verbatim}[commandchars=\\\{\}]
(15, 'hello Hello', 5.5)

    \end{Verbatim}

    \section{Classes}\label{classes}

    \begin{Verbatim}[commandchars=\\\{\}]
{\color{incolor}In [{\color{incolor}11}]:} \PY{k}{class} \PY{n+nc}{My\PYZus{}class}\PY{p}{:}
             
             \PY{k}{def} \PY{n+nf}{\PYZus{}\PYZus{}init\PYZus{}\PYZus{}}\PY{p}{(}\PY{n+nb+bp}{self}\PY{p}{,} \PY{n}{x} \PY{o}{=} \PY{l+m+mi}{10}\PY{p}{)}\PY{p}{:}
                 \PY{n+nb+bp}{self}\PY{o}{.}\PY{n}{x} \PY{o}{=} \PY{n}{x}
                 
             \PY{k}{def} \PY{n+nf}{func}\PY{p}{(}\PY{n+nb+bp}{self}\PY{p}{)}\PY{p}{:}
                 \PY{n+nb}{print}\PY{p}{(}\PY{n+nb+bp}{self}\PY{o}{.}\PY{n}{x}\PY{p}{)}
                 
         
         \PY{n}{my\PYZus{}class} \PY{o}{=} \PY{n}{My\PYZus{}class}\PY{p}{(}\PY{l+m+mi}{50}\PY{p}{)}
         
         \PY{n}{my\PYZus{}class}\PY{o}{.}\PY{n}{func}\PY{p}{(}\PY{p}{)}
\end{Verbatim}


    \begin{Verbatim}[commandchars=\\\{\}]
50

    \end{Verbatim}

    \textbf{Class inheritance:}

    \begin{Verbatim}[commandchars=\\\{\}]
{\color{incolor}In [{\color{incolor}12}]:} \PY{k}{class} \PY{n+nc}{Parent}\PY{p}{:}  
             
             \PY{k}{def} \PY{n+nf}{\PYZus{}\PYZus{}init\PYZus{}\PYZus{}}\PY{p}{(}\PY{n+nb+bp}{self}\PY{p}{)}\PY{p}{:}
                 \PY{n+nb}{print}\PY{p}{(}\PY{l+s+s2}{\PYZdq{}}\PY{l+s+s2}{Calling parent constructor}\PY{l+s+s2}{\PYZdq{}}\PY{p}{)}
                 
             \PY{k}{def} \PY{n+nf}{parentMethod}\PY{p}{(}\PY{n+nb+bp}{self}\PY{p}{)}\PY{p}{:}
                 \PY{n+nb}{print}\PY{p}{(}\PY{l+s+s1}{\PYZsq{}}\PY{l+s+s1}{Calling parent method}\PY{l+s+s1}{\PYZsq{}}\PY{p}{)}
                 
             \PY{k}{def} \PY{n+nf}{setAttr}\PY{p}{(}\PY{n+nb+bp}{self}\PY{p}{,} \PY{n}{attr}\PY{p}{)}\PY{p}{:}
                 \PY{n}{Parent}\PY{o}{.}\PY{n}{parentAttr} \PY{o}{=} \PY{n}{attr}
                 
             \PY{k}{def} \PY{n+nf}{getAttr}\PY{p}{(}\PY{n+nb+bp}{self}\PY{p}{)}\PY{p}{:}
                 \PY{n+nb}{print}\PY{p}{(}\PY{l+s+s2}{\PYZdq{}}\PY{l+s+s2}{Parent attribute :}\PY{l+s+s2}{\PYZdq{}}\PY{p}{,} \PY{n}{Parent}\PY{o}{.}\PY{n}{parentAttr}\PY{p}{)}
         
         \PY{k}{class} \PY{n+nc}{Child}\PY{p}{(}\PY{n}{Parent}\PY{p}{)}\PY{p}{:} \PY{c+c1}{\PYZsh{} define child class}
             \PY{k}{def} \PY{n+nf}{\PYZus{}\PYZus{}init\PYZus{}\PYZus{}}\PY{p}{(}\PY{n+nb+bp}{self}\PY{p}{)}\PY{p}{:}
                 \PY{n+nb}{print}\PY{p}{(}\PY{l+s+s2}{\PYZdq{}}\PY{l+s+s2}{Calling child constructor}\PY{l+s+s2}{\PYZdq{}}\PY{p}{)}
                 
             \PY{k}{def} \PY{n+nf}{childMethod}\PY{p}{(}\PY{n+nb+bp}{self}\PY{p}{)}\PY{p}{:}
                 \PY{n+nb}{print}\PY{p}{(}\PY{l+s+s1}{\PYZsq{}}\PY{l+s+s1}{Calling child method}\PY{l+s+s1}{\PYZsq{}}\PY{p}{)}
         
         \PY{n}{c} \PY{o}{=} \PY{n}{Child}\PY{p}{(}\PY{p}{)}          \PY{c+c1}{\PYZsh{} instance of child}
         \PY{n}{c}\PY{o}{.}\PY{n}{childMethod}\PY{p}{(}\PY{p}{)}      \PY{c+c1}{\PYZsh{} child calls its method}
         \PY{n}{c}\PY{o}{.}\PY{n}{parentMethod}\PY{p}{(}\PY{p}{)}     \PY{c+c1}{\PYZsh{} calls parent\PYZsq{}s method}
         \PY{n}{c}\PY{o}{.}\PY{n}{setAttr}\PY{p}{(}\PY{l+m+mi}{200}\PY{p}{)}       \PY{c+c1}{\PYZsh{} again call parent\PYZsq{}s method}
         \PY{n}{c}\PY{o}{.}\PY{n}{getAttr}\PY{p}{(}\PY{p}{)}          \PY{c+c1}{\PYZsh{} again call parent\PYZsq{}s method}
\end{Verbatim}


    \begin{Verbatim}[commandchars=\\\{\}]
Calling child constructor
Calling child method
Calling parent method
Parent attribute : 200

    \end{Verbatim}

    \section{Reading from keyboard}\label{reading-from-keyboard}

    \begin{Verbatim}[commandchars=\\\{\}]
{\color{incolor}In [{\color{incolor}13}]:} \PY{n}{x} \PY{o}{=} \PY{n+nb}{input}\PY{p}{(}\PY{l+s+s1}{\PYZsq{}}\PY{l+s+s1}{Enter your name:}\PY{l+s+s1}{\PYZsq{}}\PY{p}{)}
         \PY{n+nb}{print}\PY{p}{(}\PY{l+s+s1}{\PYZsq{}}\PY{l+s+s1}{Hello, }\PY{l+s+s1}{\PYZsq{}} \PY{o}{+} \PY{n}{x}\PY{p}{)}
\end{Verbatim}


    \begin{Verbatim}[commandchars=\\\{\}]
Enter your name:Jonas
Hello, Jonas

    \end{Verbatim}

    \section{Writing to file}\label{writing-to-file}

    \begin{Verbatim}[commandchars=\\\{\}]
{\color{incolor}In [{\color{incolor}14}]:} \PY{n}{file} \PY{o}{=} \PY{n+nb}{open}\PY{p}{(}\PY{l+s+s2}{\PYZdq{}}\PY{l+s+s2}{text.txt}\PY{l+s+s2}{\PYZdq{}}\PY{p}{,}\PY{l+s+s2}{\PYZdq{}}\PY{l+s+s2}{w}\PY{l+s+s2}{\PYZdq{}}\PY{p}{)} 
          
         \PY{n}{file}\PY{o}{.}\PY{n}{write}\PY{p}{(}\PY{l+s+s2}{\PYZdq{}}\PY{l+s+s2}{3, 4, 5,6 ,7 , 8, 9, 0, 4, 3}\PY{l+s+se}{\PYZbs{}n}\PY{l+s+s2}{\PYZdq{}}\PY{p}{)}
         \PY{n}{file}\PY{o}{.}\PY{n}{write}\PY{p}{(}\PY{l+s+s2}{\PYZdq{}}\PY{l+s+s2}{a, c, d,f ,g , h, j, k, l, i}\PY{l+s+s2}{\PYZdq{}}\PY{p}{)}
          
         \PY{n}{file}\PY{o}{.}\PY{n}{close}\PY{p}{(}\PY{p}{)} 
\end{Verbatim}


    \section{Reading from files}\label{reading-from-files}

    \begin{Verbatim}[commandchars=\\\{\}]
{\color{incolor}In [{\color{incolor}15}]:} \PY{c+c1}{\PYZsh{} Reading entire file in one go, can fail with large files}
         \PY{n}{file} \PY{o}{=} \PY{n+nb}{open}\PY{p}{(}\PY{l+s+s2}{\PYZdq{}}\PY{l+s+s2}{text.txt}\PY{l+s+s2}{\PYZdq{}}\PY{p}{,} \PY{l+s+s2}{\PYZdq{}}\PY{l+s+s2}{r}\PY{l+s+s2}{\PYZdq{}}\PY{p}{)} 
         \PY{n+nb}{print}\PY{p}{(}\PY{n}{file}\PY{o}{.}\PY{n}{read}\PY{p}{(}\PY{p}{)}\PY{p}{)}
         
         \PY{c+c1}{\PYZsh{} Read file line by line}
         \PY{n}{data} \PY{o}{=} \PY{p}{[}\PY{p}{]}
         \PY{k}{with} \PY{n+nb}{open}\PY{p}{(}\PY{l+s+s2}{\PYZdq{}}\PY{l+s+s2}{text.txt}\PY{l+s+s2}{\PYZdq{}}\PY{p}{,} \PY{l+s+s1}{\PYZsq{}}\PY{l+s+s1}{r}\PY{l+s+s1}{\PYZsq{}}\PY{p}{)} \PY{k}{as} \PY{n}{f}\PY{p}{:}
             \PY{k}{for} \PY{n}{line} \PY{o+ow}{in} \PY{n}{f}\PY{p}{:}
                 \PY{n}{new\PYZus{}instance} \PY{o}{=} \PY{n}{line}\PY{o}{.}\PY{n}{strip}\PY{p}{(}\PY{p}{)}\PY{o}{.}\PY{n}{split}\PY{p}{(}\PY{l+s+s1}{\PYZsq{}}\PY{l+s+s1}{,}\PY{l+s+s1}{\PYZsq{}}\PY{p}{)}
                 \PY{n}{data}\PY{o}{.}\PY{n}{append}\PY{p}{(}\PY{n}{new\PYZus{}instance}\PY{p}{)}
                 
         \PY{n+nb}{print}\PY{p}{(}\PY{n}{data}\PY{p}{)}
\end{Verbatim}


    \begin{Verbatim}[commandchars=\\\{\}]
3, 4, 5,6 ,7 , 8, 9, 0, 4, 3
a, c, d,f ,g , h, j, k, l, i
[['3', ' 4', ' 5', '6 ', '7 ', ' 8', ' 9', ' 0', ' 4', ' 3'], ['a', ' c', ' d', 'f ', 'g ', ' h', ' j', ' k', ' l', ' i']]

    \end{Verbatim}

    \section{Exceptions}\label{exceptions}

Exceptions help you respond to errors that can occur in your code

    \begin{Verbatim}[commandchars=\\\{\}]
{\color{incolor}In [{\color{incolor}16}]:} \PY{n}{number} \PY{o}{=} \PY{n+nb}{input}\PY{p}{(}\PY{l+s+s2}{\PYZdq{}}\PY{l+s+s2}{type an integer: }\PY{l+s+s2}{\PYZdq{}}\PY{p}{)}
         
         \PY{k}{try}\PY{p}{:}
             \PY{n}{number} \PY{o}{=} \PY{n+nb}{int}\PY{p}{(}\PY{n}{number}\PY{p}{)}
         \PY{k}{except} \PY{n+ne}{ValueError}\PY{p}{:}
             \PY{n+nb}{print}\PY{p}{(}\PY{l+s+s2}{\PYZdq{}}\PY{l+s+s2}{Invalid input}\PY{l+s+s2}{\PYZdq{}}\PY{p}{)}
         \PY{k}{else}\PY{p}{:}
             \PY{n+nb}{print}\PY{p}{(}\PY{l+s+s2}{\PYZdq{}}\PY{l+s+s2}{You did it! }\PY{l+s+s2}{\PYZdq{}} \PY{p}{,} \PY{n}{number}\PY{p}{)}
\end{Verbatim}


    \begin{Verbatim}[commandchars=\\\{\}]
type an integer: 5
You did it!  5

    \end{Verbatim}

    \section{Useful libraries}\label{useful-libraries}

    \section{Numpy}\label{numpy}

A library for large multi-dimensional arrays and matrices, along with
mathematcal functions to operate the arrays and matrices.

\textbf{\href{https://www.tutorialspoint.com/numpy/numpy_introduction.htm}{reference
1}}

\textbf{\href{https://docs.scipy.org/doc/numpy-1.15.0/user/quickstart.html\#quickstart-shape-manipulation}{reference
2}}

    \subsection{Operations using NumPy}\label{operations-using-numpy}

Using NumPy, a developer can perform the following operations − -
Mathematical and logical operations on arrays. - Fourier transforms and
routines for shape manipulation. - Operations related to linear algebra.
NumPy has in-built functions for linear algebra and random number
generation.

    \subsubsection{Environment}\label{environment}

    \begin{Verbatim}[commandchars=\\\{\}]
{\color{incolor}In [{\color{incolor}17}]:} \PY{c+c1}{\PYZsh{} Install}
         \PY{o}{!}pip install numpy
         
         \PY{c+c1}{\PYZsh{} Imprting}
         \PY{k+kn}{import} \PY{n+nn}{numpy} \PY{k}{as} \PY{n+nn}{np}
\end{Verbatim}


    \begin{Verbatim}[commandchars=\\\{\}]
Requirement already satisfied: numpy in c:\textbackslash{}users\textbackslash{}jsors\textbackslash{}anaconda3\textbackslash{}lib\textbackslash{}site-packages (1.15.4)

    \end{Verbatim}

    \subsubsection{Ndarray Object}\label{ndarray-object}

NumPy's array class is called ndarray. This object is not the same as
the standard array.array class in python. The ndarray object handles
multi-dimensional arrays, and offers more functionality. A basic numpy
array does not support \textbf{multiple data types}, for multiple arrays
with multiple data types
(\textbf{\href{https://docs.scipy.org/doc/numpy-1.10.4/user/basics.rec.html}{structured
arrays}}) can be used.

    \begin{Verbatim}[commandchars=\\\{\}]
{\color{incolor}In [{\color{incolor}18}]:} \PY{c+c1}{\PYZsh{} Create array}
         \PY{n}{a} \PY{o}{=} \PY{n}{np}\PY{o}{.}\PY{n}{array}\PY{p}{(}\PY{p}{[}\PY{l+m+mi}{1}\PY{p}{,}\PY{l+m+mi}{2}\PY{p}{,}\PY{l+m+mi}{3}\PY{p}{]}\PY{p}{)} 
         
         \PY{c+c1}{\PYZsh{} More than one dimension}
         \PY{n}{a} \PY{o}{=} \PY{n}{np}\PY{o}{.}\PY{n}{array}\PY{p}{(}\PY{p}{[}\PY{p}{[}\PY{l+m+mi}{1}\PY{p}{,} \PY{l+m+mi}{2}\PY{p}{]}\PY{p}{,} \PY{p}{[}\PY{l+m+mi}{3}\PY{p}{,} \PY{l+m+mi}{4}\PY{p}{]}\PY{p}{]}\PY{p}{)} 
         
         \PY{c+c1}{\PYZsh{} Define minimum dimensions }
         \PY{n}{a} \PY{o}{=} \PY{n}{np}\PY{o}{.}\PY{n}{array}\PY{p}{(}\PY{p}{[}\PY{l+m+mi}{1}\PY{p}{,} \PY{l+m+mi}{2}\PY{p}{,} \PY{l+m+mi}{3}\PY{p}{,} \PY{l+m+mi}{4}\PY{p}{,} \PY{l+m+mi}{5}\PY{p}{]}\PY{p}{,} \PY{n}{ndmin} \PY{o}{=} \PY{l+m+mi}{2}\PY{p}{)}
         
         \PY{c+c1}{\PYZsh{} Define data type}
         \PY{n}{a} \PY{o}{=} \PY{n}{np}\PY{o}{.}\PY{n}{array}\PY{p}{(}\PY{p}{[}\PY{l+m+mi}{1}\PY{p}{,} \PY{l+m+mi}{2}\PY{p}{,} \PY{l+m+mi}{3}\PY{p}{]}\PY{p}{,} \PY{n}{dtype} \PY{o}{=} \PY{n+nb}{complex}\PY{p}{)}
         
         \PY{n+nb}{print}\PY{p}{(}\PY{n}{a}\PY{p}{)}
\end{Verbatim}


    \begin{Verbatim}[commandchars=\\\{\}]
[1.+0.j 2.+0.j 3.+0.j]

    \end{Verbatim}

    \subsubsection{Array Creation Routines}\label{array-creation-routines}

    \begin{Verbatim}[commandchars=\\\{\}]
{\color{incolor}In [{\color{incolor}19}]:} \PY{c+c1}{\PYZsh{} zeroes fills an array with zeroes}
         \PY{n}{a} \PY{o}{=} \PY{n}{np}\PY{o}{.}\PY{n}{zeros}\PY{p}{(}\PY{p}{(}\PY{l+m+mi}{4}\PY{p}{,}\PY{l+m+mi}{5}\PY{p}{)}\PY{p}{)}
         
         \PY{c+c1}{\PYZsh{} ones fills an array with ones}
         \PY{n}{a} \PY{o}{=} \PY{n}{np}\PY{o}{.}\PY{n}{ones}\PY{p}{(}\PY{p}{(}\PY{l+m+mi}{2}\PY{p}{,}\PY{l+m+mi}{4}\PY{p}{,}\PY{l+m+mi}{6}\PY{p}{)}\PY{p}{)}
         
         \PY{c+c1}{\PYZsh{} empty fills an array with random content}
         \PY{n}{a} \PY{o}{=} \PY{n}{np}\PY{o}{.}\PY{n}{empty}\PY{p}{(}\PY{p}{(}\PY{l+m+mi}{2}\PY{p}{,}\PY{l+m+mi}{4}\PY{p}{)}\PY{p}{)}
         
         \PY{c+c1}{\PYZsh{} random fills an array with random numbers}
         \PY{n}{a} \PY{o}{=} \PY{n}{np}\PY{o}{.}\PY{n}{random}\PY{o}{.}\PY{n}{random}\PY{p}{(}\PY{p}{(}\PY{l+m+mi}{2}\PY{p}{,}\PY{l+m+mi}{3}\PY{p}{)}\PY{p}{)}\PY{o}{*} \PY{l+m+mi}{10}
         
         \PY{c+c1}{\PYZsh{} Arange fills array with a sequence}
         \PY{n}{a} \PY{o}{=} \PY{n}{np}\PY{o}{.}\PY{n}{arange}\PY{p}{(}\PY{l+m+mi}{4}\PY{p}{,} \PY{l+m+mi}{11}\PY{p}{)}
         
         \PY{c+c1}{\PYZsh{} Steps in sequence can be defined with a third parameter}
         \PY{n}{a} \PY{o}{=} \PY{n}{np}\PY{o}{.}\PY{n}{arange}\PY{p}{(}\PY{l+m+mi}{10}\PY{p}{,} \PY{l+m+mi}{25}\PY{p}{,} \PY{l+m+mi}{5}\PY{p}{)}
         \PY{n}{a} \PY{o}{=} \PY{n}{np}\PY{o}{.}\PY{n}{arange}\PY{p}{(}\PY{l+m+mi}{0}\PY{p}{,} \PY{l+m+mi}{10}\PY{p}{,} \PY{l+m+mf}{0.1}\PY{p}{)}
         
         \PY{c+c1}{\PYZsh{} linspace is similar to arange. But it uses unlike arange the}
         \PY{c+c1}{\PYZsh{} third variable to define the number of floating point variables}
         \PY{c+c1}{\PYZsh{} in the range}
         
         \PY{n}{a} \PY{o}{=} \PY{n}{np}\PY{o}{.}\PY{n}{linspace}\PY{p}{(}\PY{l+m+mi}{0}\PY{p}{,} \PY{l+m+mi}{3}\PY{p}{,} \PY{l+m+mi}{100}\PY{p}{)}
         \PY{n+nb}{print}\PY{p}{(}\PY{n}{a}\PY{p}{)}
\end{Verbatim}


    \begin{Verbatim}[commandchars=\\\{\}]
[0.         0.03030303 0.06060606 0.09090909 0.12121212 0.15151515
 0.18181818 0.21212121 0.24242424 0.27272727 0.3030303  0.33333333
 0.36363636 0.39393939 0.42424242 0.45454545 0.48484848 0.51515152
 0.54545455 0.57575758 0.60606061 0.63636364 0.66666667 0.6969697
 0.72727273 0.75757576 0.78787879 0.81818182 0.84848485 0.87878788
 0.90909091 0.93939394 0.96969697 1.         1.03030303 1.06060606
 1.09090909 1.12121212 1.15151515 1.18181818 1.21212121 1.24242424
 1.27272727 1.3030303  1.33333333 1.36363636 1.39393939 1.42424242
 1.45454545 1.48484848 1.51515152 1.54545455 1.57575758 1.60606061
 1.63636364 1.66666667 1.6969697  1.72727273 1.75757576 1.78787879
 1.81818182 1.84848485 1.87878788 1.90909091 1.93939394 1.96969697
 2.         2.03030303 2.06060606 2.09090909 2.12121212 2.15151515
 2.18181818 2.21212121 2.24242424 2.27272727 2.3030303  2.33333333
 2.36363636 2.39393939 2.42424242 2.45454545 2.48484848 2.51515152
 2.54545455 2.57575758 2.60606061 2.63636364 2.66666667 2.6969697
 2.72727273 2.75757576 2.78787879 2.81818182 2.84848485 2.87878788
 2.90909091 2.93939394 2.96969697 3.        ]

    \end{Verbatim}

    Here are some important attributes of the \textbf{ndarray} object:

    \begin{Verbatim}[commandchars=\\\{\}]
{\color{incolor}In [{\color{incolor}20}]:} \PY{n}{a} \PY{o}{=} \PY{n}{np}\PY{o}{.}\PY{n}{arange}\PY{p}{(}\PY{l+m+mi}{20}\PY{p}{)}\PY{o}{.}\PY{n}{reshape}\PY{p}{(}\PY{l+m+mi}{4}\PY{p}{,} \PY{l+m+mi}{5}\PY{p}{)}
         \PY{n+nb}{print}\PY{p}{(}\PY{n}{a}\PY{p}{)}
         
         \PY{c+c1}{\PYZsh{} Shape returns the dimensions of an array. For a 2 dimensional array}
         \PY{c+c1}{\PYZsh{} it would return a tuple with (n rows, n columns)}
         \PY{n+nb}{print}\PY{p}{(}\PY{l+s+s2}{\PYZdq{}}\PY{l+s+s2}{shape: }\PY{l+s+s2}{\PYZdq{}}\PY{p}{,} \PY{n}{a}\PY{o}{.}\PY{n}{shape}\PY{p}{)}
         
         \PY{c+c1}{\PYZsh{} ndim returns the number of dimensions in an array}
         \PY{n+nb}{print}\PY{p}{(}\PY{l+s+s2}{\PYZdq{}}\PY{l+s+s2}{ndim: }\PY{l+s+s2}{\PYZdq{}}\PY{p}{,} \PY{n}{a}\PY{o}{.}\PY{n}{ndim}\PY{p}{)}
         
         \PY{c+c1}{\PYZsh{} dtype returns the data type of the elements of an array}
         \PY{n+nb}{print}\PY{p}{(}\PY{l+s+s2}{\PYZdq{}}\PY{l+s+s2}{dtype: }\PY{l+s+s2}{\PYZdq{}}\PY{p}{,} \PY{n}{a}\PY{o}{.}\PY{n}{dtype}\PY{o}{.}\PY{n}{name}\PY{p}{)}
         
         \PY{c+c1}{\PYZsh{} The number of bytes for each element in an array}
         \PY{n+nb}{print}\PY{p}{(}\PY{l+s+s2}{\PYZdq{}}\PY{l+s+s2}{itemsize: }\PY{l+s+s2}{\PYZdq{}}\PY{p}{,} \PY{n}{a}\PY{o}{.}\PY{n}{itemsize}\PY{p}{)}
         
         \PY{c+c1}{\PYZsh{} The total number of elements in an array}
         \PY{n+nb}{print}\PY{p}{(}\PY{l+s+s2}{\PYZdq{}}\PY{l+s+s2}{size: }\PY{l+s+s2}{\PYZdq{}}\PY{p}{,} \PY{n}{a}\PY{o}{.}\PY{n}{size}\PY{p}{)}
\end{Verbatim}


    \begin{Verbatim}[commandchars=\\\{\}]
[[ 0  1  2  3  4]
 [ 5  6  7  8  9]
 [10 11 12 13 14]
 [15 16 17 18 19]]
shape:  (4, 5)
ndim:  2
dtype:  int32
itemsize:  4
size:  20

    \end{Verbatim}

    \subsubsection{Data types}\label{data-types}

Here is a list of data types that numpy supports.
(https://www.tutorialspoint.com/numpy/numpy\_data\_types.htm)

    \subsubsection{Basic Operations}\label{basic-operations}

Arithmetic operators on arrays apply elementwise. A new array is created
and filled with the result.

    \begin{Verbatim}[commandchars=\\\{\}]
{\color{incolor}In [{\color{incolor}21}]:} \PY{n}{a} \PY{o}{=} \PY{n}{np}\PY{o}{.}\PY{n}{array}\PY{p}{(} \PY{p}{[}\PY{l+m+mi}{20}\PY{p}{,}\PY{l+m+mi}{30}\PY{p}{,}\PY{l+m+mi}{40}\PY{p}{,}\PY{l+m+mi}{50}\PY{p}{]} \PY{p}{)}
         \PY{n}{b} \PY{o}{=} \PY{n}{np}\PY{o}{.}\PY{n}{array}\PY{p}{(} \PY{p}{[}\PY{l+m+mi}{0}\PY{p}{,} \PY{l+m+mi}{1}\PY{p}{,} \PY{l+m+mi}{2}\PY{p}{,} \PY{l+m+mi}{3}\PY{p}{]} \PY{p}{)}
         
         \PY{c+c1}{\PYZsh{} Addition}
         \PY{n}{c} \PY{o}{=} \PY{n}{a}\PY{o}{+}\PY{n}{b}
         
         \PY{c+c1}{\PYZsh{} Exponentation}
         \PY{n}{c} \PY{o}{=} \PY{n}{b}\PY{o}{*}\PY{o}{*}\PY{l+m+mi}{2}
         
         \PY{c+c1}{\PYZsh{} Multiplication}
         \PY{n}{c} \PY{o}{=} \PY{n}{b}\PY{o}{*}\PY{l+m+mi}{5}
         
         \PY{c+c1}{\PYZsh{} Sinus}
         \PY{n}{c} \PY{o}{=} \PY{l+m+mi}{10}\PY{o}{*}\PY{n}{np}\PY{o}{.}\PY{n}{sin}\PY{p}{(}\PY{n}{a}\PY{p}{)}
         
         \PY{c+c1}{\PYZsh{} Less than}
         \PY{n}{c} \PY{o}{=} \PY{n}{a}\PY{o}{\PYZlt{}}\PY{l+m+mi}{35}
         \PY{n+nb}{print}\PY{p}{(}\PY{n}{c}\PY{p}{)}
         
         \PY{c+c1}{\PYZsh{} Matrix product}
         \PY{n}{A} \PY{o}{=} \PY{n}{np}\PY{o}{.}\PY{n}{array}\PY{p}{(} \PY{p}{[}\PY{p}{[}\PY{l+m+mi}{1}\PY{p}{,}\PY{l+m+mi}{1}\PY{p}{]}\PY{p}{,}
                        \PY{p}{[}\PY{l+m+mi}{0}\PY{p}{,}\PY{l+m+mi}{1}\PY{p}{]}\PY{p}{]} \PY{p}{)}
         
         \PY{n}{B} \PY{o}{=} \PY{n}{np}\PY{o}{.}\PY{n}{array}\PY{p}{(} \PY{p}{[}\PY{p}{[}\PY{l+m+mi}{2}\PY{p}{,}\PY{l+m+mi}{0}\PY{p}{]}\PY{p}{,}
                        \PY{p}{[}\PY{l+m+mi}{3}\PY{p}{,}\PY{l+m+mi}{4}\PY{p}{]}\PY{p}{]} \PY{p}{)}
         \PY{n}{C} \PY{o}{=} \PY{n}{A} \PY{o}{@} \PY{n}{B}
         \PY{c+c1}{\PYZsh{} or}
         \PY{n}{C} \PY{o}{=} \PY{n}{A}\PY{o}{.}\PY{n}{dot}\PY{p}{(}\PY{n}{B}\PY{p}{)}
         
         \PY{n+nb}{print}\PY{p}{(}\PY{n}{C}\PY{p}{)}
\end{Verbatim}


    \begin{Verbatim}[commandchars=\\\{\}]
[ True  True False False]
[[5 4]
 [3 4]]

    \end{Verbatim}

    Universal functions also operate elementwise in numpy.

    \begin{Verbatim}[commandchars=\\\{\}]
{\color{incolor}In [{\color{incolor}22}]:} \PY{n}{a} \PY{o}{=} \PY{n}{np}\PY{o}{.}\PY{n}{arange}\PY{p}{(}\PY{l+m+mi}{5}\PY{p}{)}
         \PY{n+nb}{print}\PY{p}{(}\PY{n}{a}\PY{p}{)}
         \PY{n+nb}{print}\PY{p}{(}\PY{n}{np}\PY{o}{.}\PY{n}{exp}\PY{p}{(}\PY{n}{a}\PY{p}{)}\PY{p}{)}
         \PY{n+nb}{print}\PY{p}{(}\PY{n}{np}\PY{o}{.}\PY{n}{sqrt}\PY{p}{(}\PY{n}{a}\PY{p}{)}\PY{p}{)}
         \PY{n+nb}{print}\PY{p}{(}\PY{n}{np}\PY{o}{.}\PY{n}{cos}\PY{p}{(}\PY{n}{a}\PY{p}{)}\PY{p}{)}
\end{Verbatim}


    \begin{Verbatim}[commandchars=\\\{\}]
[0 1 2 3 4]
[ 1.          2.71828183  7.3890561  20.08553692 54.59815003]
[0.         1.         1.41421356 1.73205081 2.        ]
[ 1.          0.54030231 -0.41614684 -0.9899925  -0.65364362]

    \end{Verbatim}

    \subsubsection{Slicing and iterating}\label{slicing-and-iterating}

One-dimensional arrays can be indexed, sliced and iterated over, much
like lists(above) and other Python sequences.

    \begin{Verbatim}[commandchars=\\\{\}]
{\color{incolor}In [{\color{incolor}23}]:} \PY{n+nb}{print}\PY{p}{(}\PY{n}{a}\PY{p}{[}\PY{l+m+mi}{2}\PY{p}{:}\PY{l+m+mi}{4}\PY{p}{]}\PY{p}{)}
         \PY{n+nb}{print}\PY{p}{(}\PY{n}{a}\PY{p}{[} \PY{p}{:} \PY{p}{:}\PY{o}{\PYZhy{}}\PY{l+m+mi}{1}\PY{p}{]}\PY{p}{)}
\end{Verbatim}


    \begin{Verbatim}[commandchars=\\\{\}]
[2 3]
[4 3 2 1 0]

    \end{Verbatim}

    For multidimensional arrays, each dimension needs to be separated by a
comma.

    \begin{Verbatim}[commandchars=\\\{\}]
{\color{incolor}In [{\color{incolor}24}]:} \PY{n}{b} \PY{o}{=} \PY{n}{np}\PY{o}{.}\PY{n}{array}\PY{p}{(}\PY{p}{[}\PY{p}{[} \PY{l+m+mi}{0}\PY{p}{,}  \PY{l+m+mi}{1}\PY{p}{,}  \PY{l+m+mi}{2}\PY{p}{,}  \PY{l+m+mi}{3}\PY{p}{]}\PY{p}{,}
                        \PY{p}{[}\PY{l+m+mi}{10}\PY{p}{,} \PY{l+m+mi}{11}\PY{p}{,} \PY{l+m+mi}{12}\PY{p}{,} \PY{l+m+mi}{13}\PY{p}{]}\PY{p}{,}
                        \PY{p}{[}\PY{l+m+mi}{20}\PY{p}{,} \PY{l+m+mi}{21}\PY{p}{,} \PY{l+m+mi}{22}\PY{p}{,} \PY{l+m+mi}{23}\PY{p}{]}\PY{p}{,}
                        \PY{p}{[}\PY{l+m+mi}{30}\PY{p}{,} \PY{l+m+mi}{31}\PY{p}{,} \PY{l+m+mi}{32}\PY{p}{,} \PY{l+m+mi}{33}\PY{p}{]}\PY{p}{,}
                        \PY{p}{[}\PY{l+m+mi}{40}\PY{p}{,} \PY{l+m+mi}{41}\PY{p}{,} \PY{l+m+mi}{42}\PY{p}{,} \PY{l+m+mi}{43}\PY{p}{]}\PY{p}{]}\PY{p}{)}
         
         \PY{c+c1}{\PYZsh{} Indexing 3rd row and 4th column (starts at 0)}
         \PY{n+nb}{print}\PY{p}{(}\PY{n}{b}\PY{p}{[}\PY{l+m+mi}{2}\PY{p}{,}\PY{l+m+mi}{3}\PY{p}{]}\PY{p}{)}
         
         \PY{c+c1}{\PYZsh{} each row in the second column of b}
         \PY{n+nb}{print}\PY{p}{(}\PY{n}{b}\PY{p}{[}\PY{l+m+mi}{0}\PY{p}{:}\PY{p}{,} \PY{l+m+mi}{1}\PY{p}{]}\PY{p}{)}
         
         \PY{c+c1}{\PYZsh{} The first second and third elements in the first to the third }
         \PY{c+c1}{\PYZsh{} column of b}
         \PY{n+nb}{print}\PY{p}{(}\PY{n}{b}\PY{p}{[}\PY{l+m+mi}{0}\PY{p}{:}\PY{l+m+mi}{3}\PY{p}{,} \PY{l+m+mi}{1}\PY{p}{:}\PY{l+m+mi}{3}\PY{p}{]}\PY{p}{)}
         \PY{n+nb}{print}\PY{p}{(}\PY{n}{b}\PY{p}{[}\PY{l+m+mi}{3}\PY{p}{:}\PY{p}{,} \PY{l+m+mi}{2}\PY{p}{:}\PY{p}{]}\PY{p}{)}
\end{Verbatim}


    \begin{Verbatim}[commandchars=\\\{\}]
23
[ 1 11 21 31 41]
[[ 1  2]
 [11 12]
 [21 22]]
[[32 33]
 [42 43]]

    \end{Verbatim}

    Iterating over multidimensional arrays is done with respect to the first
axis:

    \begin{Verbatim}[commandchars=\\\{\}]
{\color{incolor}In [{\color{incolor}25}]:} \PY{k}{for} \PY{n}{row} \PY{o+ow}{in} \PY{n}{b}\PY{p}{:}
             \PY{n+nb}{print}\PY{p}{(}\PY{n}{row}\PY{p}{)}
             
         \PY{c+c1}{\PYZsh{} .flat can be used to flatten a multidimensional array into a}
         \PY{c+c1}{\PYZsh{} one\PYZhy{}dimensional array}
         \PY{k}{for} \PY{n}{element} \PY{o+ow}{in} \PY{n}{b}\PY{o}{.}\PY{n}{flat}\PY{p}{:}
             \PY{n+nb}{print}\PY{p}{(}\PY{n}{element}\PY{p}{)}
\end{Verbatim}


    \begin{Verbatim}[commandchars=\\\{\}]
[0 1 2 3]
[10 11 12 13]
[20 21 22 23]
[30 31 32 33]
[40 41 42 43]
0
1
2
3
10
11
12
13
20
21
22
23
30
31
32
33
40
41
42
43

    \end{Verbatim}

    \subsection{Shape manipulation}\label{shape-manipulation}

    \begin{Verbatim}[commandchars=\\\{\}]
{\color{incolor}In [{\color{incolor}26}]:} \PY{n}{a} \PY{o}{=} \PY{n}{np}\PY{o}{.}\PY{n}{floor}\PY{p}{(}\PY{n}{np}\PY{o}{.}\PY{n}{random}\PY{o}{.}\PY{n}{random}\PY{p}{(}\PY{p}{(}\PY{l+m+mi}{3}\PY{p}{,}\PY{l+m+mi}{4}\PY{p}{)}\PY{p}{)}\PY{o}{*}\PY{l+m+mi}{10}\PY{p}{)}
         \PY{n+nb}{print}\PY{p}{(}\PY{n}{a}\PY{p}{)}
         \PY{n+nb}{print}\PY{p}{(}\PY{n}{a}\PY{o}{.}\PY{n}{shape}\PY{p}{)}
         
         \PY{c+c1}{\PYZsh{} Flattens the array}
         \PY{n}{a} \PY{o}{=} \PY{n}{a}\PY{o}{.}\PY{n}{ravel}\PY{p}{(}\PY{p}{)}
         \PY{n+nb}{print}\PY{p}{(}\PY{n}{a}\PY{p}{)}
         
         \PY{c+c1}{\PYZsh{} Reshapes the array}
         \PY{n}{b} \PY{o}{=} \PY{n}{a}\PY{o}{.}\PY{n}{reshape}\PY{p}{(}\PY{l+m+mi}{3}\PY{p}{,} \PY{l+m+mi}{4}\PY{p}{)}
         \PY{n+nb}{print}\PY{p}{(}\PY{n}{b}\PY{p}{)}
         
         \PY{c+c1}{\PYZsh{} Transpose the array}
         \PY{n}{b} \PY{o}{=} \PY{n}{a}\PY{o}{.}\PY{n}{T}
         \PY{n+nb}{print}\PY{p}{(}\PY{n}{b}\PY{p}{)}
\end{Verbatim}


    \begin{Verbatim}[commandchars=\\\{\}]
[[5. 5. 9. 1.]
 [7. 2. 8. 9.]
 [6. 5. 4. 2.]]
(3, 4)
[5. 5. 9. 1. 7. 2. 8. 9. 6. 5. 4. 2.]
[[5. 5. 9. 1.]
 [7. 2. 8. 9.]
 [6. 5. 4. 2.]]
[5. 5. 9. 1. 7. 2. 8. 9. 6. 5. 4. 2.]

    \end{Verbatim}

    \section{Matplotlib}\label{matplotlib}

Data visualization library for python. Great for making visual
representations of data.

matplotlib.pyplot is a collection of command style functions that make
matplotlib work like MATLAB. Each pyplot function makes some change to a
figure: e.g., creates a figure, creates a plotting area in a figure,
plots some lines in a plotting area, decorates the plot with labels,
etc.

\textbf{\href{https://matplotlib.org/tutorials/introductory/pyplot.html}{Reference}}

    \begin{Verbatim}[commandchars=\\\{\}]
{\color{incolor}In [{\color{incolor}27}]:} \PY{c+c1}{\PYZsh{} Installation and importing}
         \PY{o}{!}pip install matplotlib
         \PY{k+kn}{import} \PY{n+nn}{matplotlib}\PY{n+nn}{.}\PY{n+nn}{pyplot} \PY{k}{as} \PY{n+nn}{plt}
         \PY{o}{\PYZpc{}}\PY{k}{matplotlib} inline 
\end{Verbatim}


    \begin{Verbatim}[commandchars=\\\{\}]
Requirement already satisfied: matplotlib in c:\textbackslash{}users\textbackslash{}jsors\textbackslash{}anaconda3\textbackslash{}lib\textbackslash{}site-packages (2.2.3)
Requirement already satisfied: pytz in c:\textbackslash{}users\textbackslash{}jsors\textbackslash{}anaconda3\textbackslash{}lib\textbackslash{}site-packages (from matplotlib) (2018.4)
Requirement already satisfied: pyparsing!=2.0.4,!=2.1.2,!=2.1.6,>=2.0.1 in c:\textbackslash{}users\textbackslash{}jsors\textbackslash{}anaconda3\textbackslash{}lib\textbackslash{}site-packages (from matplotlib) (2.2.0)
Requirement already satisfied: cycler>=0.10 in c:\textbackslash{}users\textbackslash{}jsors\textbackslash{}anaconda3\textbackslash{}lib\textbackslash{}site-packages (from matplotlib) (0.10.0)
Requirement already satisfied: kiwisolver>=1.0.1 in c:\textbackslash{}users\textbackslash{}jsors\textbackslash{}anaconda3\textbackslash{}lib\textbackslash{}site-packages (from matplotlib) (1.0.1)
Requirement already satisfied: python-dateutil>=2.1 in c:\textbackslash{}users\textbackslash{}jsors\textbackslash{}anaconda3\textbackslash{}lib\textbackslash{}site-packages (from matplotlib) (2.7.3)
Requirement already satisfied: numpy>=1.7.1 in c:\textbackslash{}users\textbackslash{}jsors\textbackslash{}anaconda3\textbackslash{}lib\textbackslash{}site-packages (from matplotlib) (1.15.4)
Requirement already satisfied: six>=1.10 in c:\textbackslash{}users\textbackslash{}jsors\textbackslash{}anaconda3\textbackslash{}lib\textbackslash{}site-packages (from matplotlib) (1.11.0)
Requirement already satisfied: setuptools in c:\textbackslash{}users\textbackslash{}jsors\textbackslash{}anaconda3\textbackslash{}lib\textbackslash{}site-packages (from kiwisolver>=1.0.1->matplotlib) (40.5.0)

    \end{Verbatim}

    \subsection{Basic plot}\label{basic-plot}

    \begin{Verbatim}[commandchars=\\\{\}]
{\color{incolor}In [{\color{incolor}28}]:} \PY{n}{y} \PY{o}{=} \PY{n}{np}\PY{o}{.}\PY{n}{arange}\PY{p}{(}\PY{l+m+mi}{6}\PY{p}{)}
         
         \PY{c+c1}{\PYZsh{} Plot array}
         \PY{n}{plt}\PY{o}{.}\PY{n}{plot}\PY{p}{(}\PY{n}{y}\PY{p}{)}
         \PY{c+c1}{\PYZsh{} Set labels}
         \PY{n}{plt}\PY{o}{.}\PY{n}{ylabel}\PY{p}{(}\PY{l+s+s2}{\PYZdq{}}\PY{l+s+s2}{Numbers}\PY{l+s+s2}{\PYZdq{}}\PY{p}{)} \PY{c+c1}{\PYZsh{} y label}
         \PY{n}{plt}\PY{o}{.}\PY{n}{xlabel}\PY{p}{(}\PY{l+s+s2}{\PYZdq{}}\PY{l+s+s2}{Autofilled sequence}\PY{l+s+s2}{\PYZdq{}}\PY{p}{)} \PY{c+c1}{\PYZsh{} x label}
         \PY{n}{plt}\PY{o}{.}\PY{n}{show}\PY{p}{(}\PY{p}{)}
\end{Verbatim}


    \begin{center}
    \adjustimage{max size={0.9\linewidth}{0.9\paperheight}}{output_62_0.png}
    \end{center}
    { \hspace*{\fill} \\}
    
    As you can see, if only a single array is passed to plot(), matplotlib
will automatically generate the x-values.

The x-values can be defined by passing a second array.

    \begin{Verbatim}[commandchars=\\\{\}]
{\color{incolor}In [{\color{incolor}29}]:} \PY{n}{x} \PY{o}{=} \PY{p}{[}\PY{l+m+mi}{0}\PY{p}{,} \PY{l+m+mi}{10}\PY{p}{,} \PY{l+m+mi}{20}\PY{p}{,} \PY{l+m+mi}{30}\PY{p}{,} \PY{l+m+mi}{40}\PY{p}{,} \PY{l+m+mi}{50}\PY{p}{]}
         \PY{n}{plt}\PY{o}{.}\PY{n}{plot}\PY{p}{(}\PY{n}{x}\PY{p}{,}\PY{n}{y}\PY{p}{)}
         \PY{n}{plt}\PY{o}{.}\PY{n}{show}\PY{p}{(}\PY{p}{)}
\end{Verbatim}


    \begin{center}
    \adjustimage{max size={0.9\linewidth}{0.9\paperheight}}{output_64_0.png}
    \end{center}
    { \hspace*{\fill} \\}
    
    \subsubsection{Labels and ticks}\label{labels-and-ticks}

How to label different parts of the plot, add text, and change precision
of axes:

    \begin{Verbatim}[commandchars=\\\{\}]
{\color{incolor}In [{\color{incolor}30}]:} \PY{n}{a} \PY{o}{=} \PY{n}{np}\PY{o}{.}\PY{n}{random}\PY{o}{.}\PY{n}{random}\PY{p}{(}\PY{l+m+mi}{20}\PY{p}{)}\PY{o}{*}\PY{l+m+mi}{10}
         \PY{n}{ticks} \PY{o}{=} \PY{n}{np}\PY{o}{.}\PY{n}{arange}\PY{p}{(}\PY{l+m+mi}{0}\PY{p}{,} \PY{l+m+mi}{11}\PY{p}{,} \PY{l+m+mf}{0.5}\PY{p}{)}
         \PY{n+nb}{print}\PY{p}{(}\PY{n}{ticks}\PY{p}{)}
         
         \PY{n}{plt}\PY{o}{.}\PY{n}{title}\PY{p}{(}\PY{l+s+s1}{\PYZsq{}}\PY{l+s+s1}{Title}\PY{l+s+s1}{\PYZsq{}}\PY{p}{,} \PY{n}{color}\PY{o}{=}\PY{l+s+s1}{\PYZsq{}}\PY{l+s+s1}{red}\PY{l+s+s1}{\PYZsq{}}\PY{p}{,} \PY{n}{fontsize}\PY{o}{=}\PY{l+m+mi}{20}\PY{p}{)}
         \PY{n}{plt}\PY{o}{.}\PY{n}{xlabel}\PY{p}{(}\PY{l+s+s1}{\PYZsq{}}\PY{l+s+s1}{X label}\PY{l+s+s1}{\PYZsq{}}\PY{p}{,} \PY{n}{color}\PY{o}{=}\PY{p}{(}\PY{l+m+mf}{0.3}\PY{p}{,} \PY{l+m+mf}{0.1}\PY{p}{,} \PY{l+m+mf}{0.9}\PY{p}{)}\PY{p}{,} \PY{n}{fontsize}\PY{o}{=}\PY{l+m+mi}{30}\PY{p}{)}
         \PY{n}{plt}\PY{o}{.}\PY{n}{ylabel}\PY{p}{(}\PY{l+s+s1}{\PYZsq{}}\PY{l+s+s1}{Y label}\PY{l+s+s1}{\PYZsq{}}\PY{p}{)}
         \PY{n}{plt}\PY{o}{.}\PY{n}{yticks}\PY{p}{(}\PY{n}{ticks}\PY{p}{)} \PY{c+c1}{\PYZsh{} precision of y axis}
         
         \PY{c+c1}{\PYZsh{} Add some text}
         \PY{n}{plt}\PY{o}{.}\PY{n}{text}\PY{p}{(}\PY{l+m+mi}{14}\PY{p}{,} \PY{l+m+mi}{9}\PY{p}{,} \PY{l+s+s1}{\PYZsq{}}\PY{l+s+s1}{Green text}\PY{l+s+s1}{\PYZsq{}}\PY{p}{,} \PY{n}{color}\PY{o}{=}\PY{l+s+s1}{\PYZsq{}}\PY{l+s+s1}{green}\PY{l+s+s1}{\PYZsq{}}\PY{p}{,} \PY{n}{fontsize}\PY{o}{=}\PY{l+m+mi}{16}\PY{p}{)}
         
         \PY{c+c1}{\PYZsh{} Create annotation}
         \PY{n}{plt}\PY{o}{.}\PY{n}{annotate}\PY{p}{(}\PY{l+s+s1}{\PYZsq{}}\PY{l+s+s1}{Look at this spot}\PY{l+s+s1}{\PYZsq{}}\PY{p}{,} \PY{n}{xy}\PY{o}{=}\PY{p}{(}\PY{l+m+mi}{4}\PY{p}{,} \PY{n}{a}\PY{p}{[}\PY{l+m+mi}{4}\PY{p}{]}\PY{p}{)}\PY{p}{,} \PY{n}{xytext}\PY{o}{=}\PY{p}{(}\PY{l+m+mi}{3}\PY{p}{,} \PY{n}{a}\PY{p}{[}\PY{l+m+mi}{4}\PY{p}{]}\PY{o}{+}\PY{l+m+mi}{3}\PY{p}{)}\PY{p}{,}
                      \PY{n}{arrowprops}\PY{o}{=}\PY{n+nb}{dict}\PY{p}{(}\PY{n}{facecolor}\PY{o}{=}\PY{l+s+s1}{\PYZsq{}}\PY{l+s+s1}{black}\PY{l+s+s1}{\PYZsq{}}\PY{p}{)}\PY{p}{,}\PY{p}{)}
         \PY{n}{plt}\PY{o}{.}\PY{n}{plot}\PY{p}{(}\PY{n}{a}\PY{p}{)}
         \PY{n}{plt}\PY{o}{.}\PY{n}{show}\PY{p}{(}\PY{p}{)}
\end{Verbatim}


    \begin{Verbatim}[commandchars=\\\{\}]
[ 0.   0.5  1.   1.5  2.   2.5  3.   3.5  4.   4.5  5.   5.5  6.   6.5
  7.   7.5  8.   8.5  9.   9.5 10.  10.5]

    \end{Verbatim}

    \begin{center}
    \adjustimage{max size={0.9\linewidth}{0.9\paperheight}}{output_66_1.png}
    \end{center}
    { \hspace*{\fill} \\}
    
    \subsection{Formatting}\label{formatting}

    In the plot() parameters you can add a third parameter with a formatting
string which specifies the color and line type of the plot.

You can use the axis({[}xmin, xmax, ymin, ymax{]}) command to specify
the viewport of the axes.

    \begin{Verbatim}[commandchars=\\\{\}]
{\color{incolor}In [{\color{incolor}31}]:} \PY{c+c1}{\PYZsh{} The third parameter \PYZsq{}ro\PYZsq{} spesifies that the style should be }
         \PY{c+c1}{\PYZsh{} red and oval}
         \PY{n}{plt}\PY{o}{.}\PY{n}{plot}\PY{p}{(}\PY{p}{[}\PY{l+m+mi}{1}\PY{p}{,} \PY{l+m+mi}{2}\PY{p}{,} \PY{l+m+mi}{3}\PY{p}{,} \PY{l+m+mi}{4}\PY{p}{]}\PY{p}{,} \PY{p}{[}\PY{l+m+mi}{1}\PY{p}{,} \PY{l+m+mi}{4}\PY{p}{,} \PY{l+m+mi}{9}\PY{p}{,} \PY{l+m+mi}{16}\PY{p}{]}\PY{p}{,} \PY{l+s+s1}{\PYZsq{}}\PY{l+s+s1}{go}\PY{l+s+s1}{\PYZsq{}}\PY{p}{)}
         \PY{n}{plt}\PY{o}{.}\PY{n}{axis}\PY{p}{(}\PY{p}{[}\PY{l+m+mi}{0}\PY{p}{,} \PY{l+m+mi}{6}\PY{p}{,} \PY{l+m+mi}{0}\PY{p}{,} \PY{l+m+mi}{20}\PY{p}{]}\PY{p}{)}
         \PY{n}{plt}\PY{o}{.}\PY{n}{show}\PY{p}{(}\PY{p}{)}
         
         \PY{c+c1}{\PYZsh{} The color can alsobe specified with RBG, RGBA or hex strings}
         \PY{n}{plt}\PY{o}{.}\PY{n}{plot}\PY{p}{(}\PY{p}{[}\PY{l+m+mi}{1}\PY{p}{,} \PY{l+m+mi}{2}\PY{p}{,} \PY{l+m+mi}{3}\PY{p}{,} \PY{l+m+mi}{4}\PY{p}{]}\PY{p}{,} \PY{p}{[}\PY{l+m+mi}{1}\PY{p}{,} \PY{l+m+mi}{4}\PY{p}{,} \PY{l+m+mi}{9}\PY{p}{,} \PY{l+m+mi}{16}\PY{p}{]}\PY{p}{,} \PY{l+s+s1}{\PYZsq{}}\PY{l+s+s1}{o}\PY{l+s+s1}{\PYZsq{}}\PY{p}{,} \PY{n}{color}\PY{o}{=}\PY{p}{(}\PY{l+m+mf}{0.4}\PY{p}{,}\PY{l+m+mf}{0.3}\PY{p}{,}\PY{l+m+mf}{0.8}\PY{p}{)}\PY{p}{)}
         \PY{n}{plt}\PY{o}{.}\PY{n}{axis}\PY{p}{(}\PY{p}{[}\PY{l+m+mi}{0}\PY{p}{,} \PY{l+m+mi}{20}\PY{p}{,} \PY{l+m+mi}{0}\PY{p}{,} \PY{l+m+mi}{40}\PY{p}{]}\PY{p}{)}
         \PY{n}{plt}\PY{o}{.}\PY{n}{show}\PY{p}{(}\PY{p}{)}
\end{Verbatim}


    \begin{center}
    \adjustimage{max size={0.9\linewidth}{0.9\paperheight}}{output_69_0.png}
    \end{center}
    { \hspace*{\fill} \\}
    
    \begin{center}
    \adjustimage{max size={0.9\linewidth}{0.9\paperheight}}{output_69_1.png}
    \end{center}
    { \hspace*{\fill} \\}
    
    Here are some markers, styles and colors that matplotlib offers:

    \begin{Verbatim}[commandchars=\\\{\}]
{\color{incolor}In [{\color{incolor}32}]:} \PY{n}{filled\PYZus{}markers} \PY{o}{=} \PY{p}{(}\PY{l+s+s1}{\PYZsq{}}\PY{l+s+s1}{o}\PY{l+s+s1}{\PYZsq{}}\PY{p}{,} \PY{l+s+s1}{\PYZsq{}}\PY{l+s+s1}{v}\PY{l+s+s1}{\PYZsq{}}\PY{p}{,} \PY{l+s+s1}{\PYZsq{}}\PY{l+s+s1}{\PYZca{}}\PY{l+s+s1}{\PYZsq{}}\PY{p}{,} \PY{l+s+s1}{\PYZsq{}}\PY{l+s+s1}{\PYZlt{}}\PY{l+s+s1}{\PYZsq{}}\PY{p}{,} \PY{l+s+s1}{\PYZsq{}}\PY{l+s+s1}{\PYZgt{}}\PY{l+s+s1}{\PYZsq{}}\PY{p}{,} \PY{l+s+s1}{\PYZsq{}}\PY{l+s+s1}{8}\PY{l+s+s1}{\PYZsq{}}\PY{p}{,} \PY{l+s+s1}{\PYZsq{}}\PY{l+s+s1}{s}\PY{l+s+s1}{\PYZsq{}}\PY{p}{,} 
                           \PY{l+s+s1}{\PYZsq{}}\PY{l+s+s1}{p}\PY{l+s+s1}{\PYZsq{}}\PY{p}{,} \PY{l+s+s1}{\PYZsq{}}\PY{l+s+s1}{*}\PY{l+s+s1}{\PYZsq{}}\PY{p}{,} \PY{l+s+s1}{\PYZsq{}}\PY{l+s+s1}{h}\PY{l+s+s1}{\PYZsq{}}\PY{p}{,} \PY{l+s+s1}{\PYZsq{}}\PY{l+s+s1}{H}\PY{l+s+s1}{\PYZsq{}}\PY{p}{,} \PY{l+s+s1}{\PYZsq{}}\PY{l+s+s1}{D}\PY{l+s+s1}{\PYZsq{}}\PY{p}{,} \PY{l+s+s1}{\PYZsq{}}\PY{l+s+s1}{d}\PY{l+s+s1}{\PYZsq{}}\PY{p}{,} \PY{l+s+s1}{\PYZsq{}}\PY{l+s+s1}{P}\PY{l+s+s1}{\PYZsq{}}\PY{p}{,} \PY{l+s+s1}{\PYZsq{}}\PY{l+s+s1}{X}\PY{l+s+s1}{\PYZsq{}}\PY{p}{)}
         \PY{n}{line\PYZus{}styles} \PY{o}{=} \PY{p}{(}\PY{l+s+s1}{\PYZsq{}}\PY{l+s+s1}{:}\PY{l+s+s1}{\PYZsq{}}\PY{p}{,} \PY{l+s+s1}{\PYZsq{}}\PY{l+s+s1}{\PYZhy{},}\PY{l+s+s1}{\PYZsq{}}\PY{p}{,} \PY{l+s+s1}{\PYZsq{}}\PY{l+s+s1}{\PYZhy{}\PYZhy{}}\PY{l+s+s1}{\PYZsq{}}\PY{p}{,} \PY{l+s+s1}{\PYZsq{}}\PY{l+s+s1}{\PYZhy{}}\PY{l+s+s1}{\PYZsq{}}\PY{p}{)}
         \PY{n}{line\PYZus{}colors} \PY{o}{=} \PY{p}{(}\PY{l+s+s1}{\PYZsq{}}\PY{l+s+s1}{b}\PY{l+s+s1}{\PYZsq{}}\PY{p}{,} \PY{l+s+s1}{\PYZsq{}}\PY{l+s+s1}{g}\PY{l+s+s1}{\PYZsq{}}\PY{p}{,} \PY{l+s+s1}{\PYZsq{}}\PY{l+s+s1}{r}\PY{l+s+s1}{\PYZsq{}}\PY{p}{,} \PY{l+s+s1}{\PYZsq{}}\PY{l+s+s1}{c}\PY{l+s+s1}{\PYZsq{}}\PY{p}{,} \PY{l+s+s1}{\PYZsq{}}\PY{l+s+s1}{m}\PY{l+s+s1}{\PYZsq{}}\PY{p}{,} \PY{l+s+s1}{\PYZsq{}}\PY{l+s+s1}{y}\PY{l+s+s1}{\PYZsq{}}\PY{p}{,} \PY{l+s+s1}{\PYZsq{}}\PY{l+s+s1}{k}\PY{l+s+s1}{\PYZsq{}}\PY{p}{,} \PY{l+s+s1}{\PYZsq{}}\PY{l+s+s1}{w}\PY{l+s+s1}{\PYZsq{}}\PY{p}{)}
         
         \PY{n}{y} \PY{o}{=} \PY{p}{[}\PY{l+m+mi}{3}\PY{p}{,} \PY{l+m+mi}{16}\PY{p}{,} \PY{l+m+mi}{14}\PY{p}{,} \PY{l+m+mi}{26}\PY{p}{,} \PY{l+m+mi}{32}\PY{p}{,} \PY{l+m+mi}{44}\PY{p}{,} \PY{l+m+mi}{5}\PY{p}{,} \PY{l+m+mi}{8}\PY{p}{,} \PY{l+m+mi}{10}\PY{p}{]}
         \PY{n}{plt}\PY{o}{.}\PY{n}{plot}\PY{p}{(}\PY{n}{y}\PY{p}{,} \PY{n}{linestyle}\PY{o}{=}\PY{n}{line\PYZus{}styles}\PY{p}{[}\PY{l+m+mi}{2}\PY{p}{]}\PY{p}{,} \PY{n}{color}\PY{o}{=}\PY{n}{line\PYZus{}colors}\PY{p}{[}\PY{l+m+mi}{4}\PY{p}{]}\PY{p}{,} \PY{n}{marker}\PY{o}{=}\PY{n}{filled\PYZus{}markers}\PY{p}{[}\PY{l+m+mi}{5}\PY{p}{]}\PY{p}{)}
         \PY{n}{plt}\PY{o}{.}\PY{n}{show}\PY{p}{(}\PY{p}{)}
\end{Verbatim}


    \begin{center}
    \adjustimage{max size={0.9\linewidth}{0.9\paperheight}}{output_71_0.png}
    \end{center}
    { \hspace*{\fill} \\}
    
    Serveral lists with different styles can be plotted at the same time:

    \begin{Verbatim}[commandchars=\\\{\}]
{\color{incolor}In [{\color{incolor}33}]:} \PY{n}{a} \PY{o}{=} \PY{n}{np}\PY{o}{.}\PY{n}{arange}\PY{p}{(}\PY{l+m+mi}{0}\PY{p}{,} \PY{l+m+mi}{3}\PY{p}{,} \PY{l+m+mf}{0.1}\PY{p}{)}
         \PY{n}{b} \PY{o}{=} \PY{n}{a}\PY{p}{[}\PY{p}{:}\PY{p}{]} \PY{o}{*}\PY{o}{*}\PY{l+m+mi}{2}
         \PY{n}{c} \PY{o}{=} \PY{n}{a}\PY{p}{[}\PY{p}{:}\PY{p}{]} \PY{o}{*}\PY{o}{*}\PY{l+m+mi}{3}
         
         \PY{c+c1}{\PYZsh{} red dashes, blue squares and green triangles}
         \PY{n}{plt}\PY{o}{.}\PY{n}{plot}\PY{p}{(}\PY{n}{a}\PY{p}{,} \PY{l+s+s1}{\PYZsq{}}\PY{l+s+s1}{r\PYZhy{}\PYZhy{}}\PY{l+s+s1}{\PYZsq{}}\PY{p}{,}
                  \PY{n}{b}\PY{p}{,} \PY{l+s+s1}{\PYZsq{}}\PY{l+s+s1}{bs}\PY{l+s+s1}{\PYZsq{}}\PY{p}{,} 
                  \PY{n}{c}\PY{p}{,} \PY{l+s+s1}{\PYZsq{}}\PY{l+s+s1}{g\PYZca{}}\PY{l+s+s1}{\PYZsq{}}\PY{p}{)}
         \PY{n}{plt}\PY{o}{.}\PY{n}{show}\PY{p}{(}\PY{p}{)}
\end{Verbatim}


    \begin{center}
    \adjustimage{max size={0.9\linewidth}{0.9\paperheight}}{output_73_0.png}
    \end{center}
    { \hspace*{\fill} \\}
    
    \subsubsection{Subplots}\label{subplots}

Multiple plots can be added to the same figure:

    \begin{Verbatim}[commandchars=\\\{\}]
{\color{incolor}In [{\color{incolor}34}]:} \PY{c+c1}{\PYZsh{} Creates some random arrays of different sizes and value ranges}
         \PY{n}{a1} \PY{o}{=} \PY{n}{np}\PY{o}{.}\PY{n}{random}\PY{o}{.}\PY{n}{random}\PY{p}{(}\PY{l+m+mi}{20}\PY{p}{)}
         \PY{n}{a2} \PY{o}{=} \PY{n}{np}\PY{o}{.}\PY{n}{random}\PY{o}{.}\PY{n}{random}\PY{p}{(}\PY{l+m+mi}{30}\PY{p}{)}\PY{o}{*}\PY{l+m+mi}{2}
         \PY{n}{a3} \PY{o}{=} \PY{n}{np}\PY{o}{.}\PY{n}{random}\PY{o}{.}\PY{n}{random}\PY{p}{(}\PY{l+m+mi}{40}\PY{p}{)}\PY{o}{*}\PY{l+m+mi}{4}
         \PY{n}{a4} \PY{o}{=} \PY{n}{np}\PY{o}{.}\PY{n}{random}\PY{o}{.}\PY{n}{random}\PY{p}{(}\PY{l+m+mi}{50}\PY{p}{)}\PY{o}{*}\PY{l+m+mi}{8}
         
         \PY{c+c1}{\PYZsh{} Here we create 4 subplots with plt.subplots(numrows, numcolumns, ...)}
         \PY{c+c1}{\PYZsh{} plt.subplots() returns the subplots and a fig object(the entire figure)}
         \PY{n}{fig}\PY{p}{,} \PY{p}{(}\PY{p}{(}\PY{n}{ax1}\PY{p}{,} \PY{n}{ax2}\PY{p}{)}\PY{p}{,} \PY{p}{(}\PY{n}{ax3}\PY{p}{,} \PY{n}{ax4}\PY{p}{)}\PY{p}{)} \PY{o}{=} \PY{n}{plt}\PY{o}{.}\PY{n}{subplots}\PY{p}{(}\PY{l+m+mi}{2}\PY{p}{,} \PY{l+m+mi}{2}\PY{p}{,} \PY{n}{sharex}\PY{o}{=}\PY{k+kc}{False}\PY{p}{,} \PY{n}{sharey}\PY{o}{=}\PY{k+kc}{False}\PY{p}{)}
         \PY{n}{ax1}\PY{o}{.}\PY{n}{plot}\PY{p}{(}\PY{n}{a1}\PY{p}{)}
         \PY{n}{ax2}\PY{o}{.}\PY{n}{plot}\PY{p}{(}\PY{n}{a2}\PY{p}{)}
         \PY{n}{ax3}\PY{o}{.}\PY{n}{plot}\PY{p}{(}\PY{n}{a3}\PY{p}{)}
         \PY{n}{ax4}\PY{o}{.}\PY{n}{plot}\PY{p}{(}\PY{n}{a4}\PY{p}{)}
\end{Verbatim}


\begin{Verbatim}[commandchars=\\\{\}]
{\color{outcolor}Out[{\color{outcolor}34}]:} [<matplotlib.lines.Line2D at 0x1e0321f93c8>]
\end{Verbatim}
            
    \begin{center}
    \adjustimage{max size={0.9\linewidth}{0.9\paperheight}}{output_75_1.png}
    \end{center}
    { \hspace*{\fill} \\}
    
    \subsubsection{Different ways to visualize
data}\label{different-ways-to-visualize-data}

Matplotlib isn't only able to plot lines. It can be used to visualize a
variety of data, from pie charts to images. For subplots you have to
call set\_ functions, plt.set\_title() instead of plt.title() for
example. Here are some examples:

    \begin{Verbatim}[commandchars=\\\{\}]
{\color{incolor}In [{\color{incolor}35}]:} \PY{n}{fig}\PY{p}{,} \PY{p}{(}\PY{p}{(}\PY{n}{bar\PYZus{}plot}\PY{p}{,} \PY{n}{pie\PYZus{}chart}\PY{p}{)}\PY{p}{,} \PY{p}{(}\PY{n}{scatter\PYZus{}plot}\PY{p}{,} \PY{n}{image\PYZus{}plot}\PY{p}{)}\PY{p}{)} \PY{o}{=} \PY{n}{plt}\PY{o}{.}\PY{n}{subplots}\PY{p}{(}\PY{l+m+mi}{2}\PY{p}{,} \PY{l+m+mi}{2}\PY{p}{,} \PY{n}{sharex}\PY{o}{=}\PY{k+kc}{False}\PY{p}{,} \PY{n}{sharey}\PY{o}{=}\PY{k+kc}{False}\PY{p}{)}
         
         \PY{c+c1}{\PYZsh{} Bar plot}
         \PY{n}{bar\PYZus{}data} \PY{o}{=} \PY{n}{np}\PY{o}{.}\PY{n}{random}\PY{o}{.}\PY{n}{random}\PY{p}{(}\PY{l+m+mi}{5}\PY{p}{)}
         \PY{n}{x} \PY{o}{=} \PY{n}{np}\PY{o}{.}\PY{n}{arange}\PY{p}{(}\PY{l+m+mi}{5}\PY{p}{)}
         \PY{n}{bar\PYZus{}plot}\PY{o}{.}\PY{n}{set\PYZus{}title}\PY{p}{(}\PY{l+s+s1}{\PYZsq{}}\PY{l+s+s1}{Bar plot}\PY{l+s+s1}{\PYZsq{}}\PY{p}{)}
         \PY{n}{bar\PYZus{}plot}\PY{o}{.}\PY{n}{bar}\PY{p}{(}\PY{n}{x}\PY{p}{,}\PY{n}{bar\PYZus{}data}\PY{p}{)}
         
         \PY{c+c1}{\PYZsh{}Pie chart}
         \PY{n}{labels} \PY{o}{=} \PY{l+s+s1}{\PYZsq{}}\PY{l+s+s1}{Frogs}\PY{l+s+s1}{\PYZsq{}}\PY{p}{,} \PY{l+s+s1}{\PYZsq{}}\PY{l+s+s1}{Hogs}\PY{l+s+s1}{\PYZsq{}}\PY{p}{,} \PY{l+s+s1}{\PYZsq{}}\PY{l+s+s1}{Dogs}\PY{l+s+s1}{\PYZsq{}}\PY{p}{,} \PY{l+s+s1}{\PYZsq{}}\PY{l+s+s1}{Logs}\PY{l+s+s1}{\PYZsq{}}
         \PY{n}{sizes} \PY{o}{=} \PY{p}{[}\PY{l+m+mi}{15}\PY{p}{,} \PY{l+m+mi}{30}\PY{p}{,} \PY{l+m+mi}{45}\PY{p}{,} \PY{l+m+mi}{10}\PY{p}{]}
         \PY{n}{explode} \PY{o}{=} \PY{p}{(}\PY{l+m+mf}{0.0}\PY{p}{,} \PY{l+m+mf}{0.1}\PY{p}{,} \PY{l+m+mi}{0}\PY{p}{,} \PY{l+m+mi}{0}\PY{p}{)}  \PY{c+c1}{\PYZsh{} Extrudes the \PYZsq{}Hogs\PYZsq{} slice}
         
         \PY{n}{pie\PYZus{}chart}\PY{o}{.}\PY{n}{set\PYZus{}title}\PY{p}{(}\PY{l+s+s2}{\PYZdq{}}\PY{l+s+s2}{Pie chart}\PY{l+s+s2}{\PYZdq{}}\PY{p}{)}
         \PY{n}{pie\PYZus{}chart}\PY{o}{.}\PY{n}{pie}\PY{p}{(}\PY{n}{sizes}\PY{p}{,} \PY{n}{explode}\PY{o}{=}\PY{n}{explode}\PY{p}{,} \PY{n}{labels}\PY{o}{=}\PY{n}{labels}\PY{p}{,} \PY{n}{autopct}\PY{o}{=}\PY{l+s+s1}{\PYZsq{}}\PY{l+s+si}{\PYZpc{}0d}\PY{l+s+si}{\PYZpc{}\PYZpc{}}\PY{l+s+s1}{\PYZsq{}}\PY{p}{,}
                 \PY{n}{shadow}\PY{o}{=}\PY{k+kc}{True}\PY{p}{,} \PY{n}{startangle}\PY{o}{=}\PY{l+m+mi}{90}\PY{p}{)}
         \PY{n}{pie\PYZus{}chart}\PY{o}{.}\PY{n}{axis}\PY{p}{(}\PY{l+s+s1}{\PYZsq{}}\PY{l+s+s1}{equal}\PY{l+s+s1}{\PYZsq{}}\PY{p}{)}  \PY{c+c1}{\PYZsh{} Ensures chart is drawn as a circle}
         
         \PY{c+c1}{\PYZsh{} Scatter plot}
         \PY{n}{x} \PY{o}{=} \PY{n}{np}\PY{o}{.}\PY{n}{random}\PY{o}{.}\PY{n}{rand}\PY{p}{(}\PY{l+m+mi}{40}\PY{p}{)}
         \PY{n}{y} \PY{o}{=} \PY{n}{np}\PY{o}{.}\PY{n}{random}\PY{o}{.}\PY{n}{rand}\PY{p}{(}\PY{l+m+mi}{40}\PY{p}{)}
         \PY{n}{scatter\PYZus{}plot}\PY{o}{.}\PY{n}{set\PYZus{}title}\PY{p}{(}\PY{l+s+s1}{\PYZsq{}}\PY{l+s+s1}{Scatter plot}\PY{l+s+s1}{\PYZsq{}}\PY{p}{)}
         \PY{n}{scatter\PYZus{}plot}\PY{o}{.}\PY{n}{scatter}\PY{p}{(}\PY{n}{x}\PY{p}{,} \PY{n}{y}\PY{p}{,} \PY{n}{color}\PY{o}{=}\PY{l+s+s1}{\PYZsq{}}\PY{l+s+s1}{red}\PY{l+s+s1}{\PYZsq{}}\PY{p}{)}
         
         \PY{c+c1}{\PYZsh{} Image}
         \PY{k+kn}{import} \PY{n+nn}{matplotlib}\PY{n+nn}{.}\PY{n+nn}{image} \PY{k}{as} \PY{n+nn}{mpimg}
         \PY{n}{img}\PY{o}{=}\PY{n}{mpimg}\PY{o}{.}\PY{n}{imread}\PY{p}{(}\PY{l+s+s1}{\PYZsq{}}\PY{l+s+s1}{dog.jpg}\PY{l+s+s1}{\PYZsq{}}\PY{p}{)}
         \PY{n}{image\PYZus{}plot}\PY{o}{.}\PY{n}{imshow}\PY{p}{(}\PY{n}{img}\PY{p}{)}
\end{Verbatim}


\begin{Verbatim}[commandchars=\\\{\}]
{\color{outcolor}Out[{\color{outcolor}35}]:} <matplotlib.image.AxesImage at 0x1e033336a20>
\end{Verbatim}
            
    \begin{center}
    \adjustimage{max size={0.9\linewidth}{0.9\paperheight}}{output_77_1.png}
    \end{center}
    { \hspace*{\fill} \\}
    
    This overview is just scratching the surface of what you can you in
matplotlib. Luckily it is a very well documented library, with examples
for just about everything. Go
\textbf{\href{https://matplotlib.org/api/pyplot_summary.html}{here}} to
see an overview of the pyplot functions.

    \section{Pandas}\label{pandas}

Library providing high-performance, easy-to-use data structures and data
analysis tools for Python. Pandas runs on top of numpy, and is very
popular for data science.

    \subsubsection{Installation}\label{installation}

    \begin{Verbatim}[commandchars=\\\{\}]
{\color{incolor}In [{\color{incolor}36}]:} \PY{o}{!}pip install pandas
         
         \PY{k+kn}{import} \PY{n+nn}{pandas} \PY{k}{as} \PY{n+nn}{pd}
\end{Verbatim}


    \begin{Verbatim}[commandchars=\\\{\}]
Requirement already satisfied: pandas in c:\textbackslash{}users\textbackslash{}jsors\textbackslash{}anaconda3\textbackslash{}lib\textbackslash{}site-packages (0.23.4)
Requirement already satisfied: numpy>=1.9.0 in c:\textbackslash{}users\textbackslash{}jsors\textbackslash{}anaconda3\textbackslash{}lib\textbackslash{}site-packages (from pandas) (1.15.4)
Requirement already satisfied: pytz>=2011k in c:\textbackslash{}users\textbackslash{}jsors\textbackslash{}anaconda3\textbackslash{}lib\textbackslash{}site-packages (from pandas) (2018.4)
Requirement already satisfied: python-dateutil>=2.5.0 in c:\textbackslash{}users\textbackslash{}jsors\textbackslash{}anaconda3\textbackslash{}lib\textbackslash{}site-packages (from pandas) (2.7.3)
Requirement already satisfied: six>=1.5 in c:\textbackslash{}users\textbackslash{}jsors\textbackslash{}anaconda3\textbackslash{}lib\textbackslash{}site-packages (from python-dateutil>=2.5.0->pandas) (1.11.0)

    \end{Verbatim}

    \subsubsection{Reading data}\label{reading-data}

    \begin{Verbatim}[commandchars=\\\{\}]
{\color{incolor}In [{\color{incolor}37}]:} \PY{n}{df} \PY{o}{=} \PY{n}{pd}\PY{o}{.}\PY{n}{read\PYZus{}csv}\PY{p}{(}\PY{l+s+s1}{\PYZsq{}}\PY{l+s+s1}{winequality\PYZhy{}white.csv}\PY{l+s+s1}{\PYZsq{}}\PY{p}{,} \PY{n}{delimiter} \PY{o}{=} \PY{l+s+s1}{\PYZsq{}}\PY{l+s+s1}{;}\PY{l+s+s1}{\PYZsq{}}\PY{p}{)}
         
         \PY{c+c1}{\PYZsh{} Prints the first 10 rows}
         \PY{n}{df}\PY{o}{.}\PY{n}{head}\PY{p}{(}\PY{l+m+mi}{10}\PY{p}{)}
\end{Verbatim}


\begin{Verbatim}[commandchars=\\\{\}]
{\color{outcolor}Out[{\color{outcolor}37}]:}    fixed acidity  volatile acidity  citric acid  residual sugar  chlorides  \textbackslash{}
         0            7.0              0.27         0.36            20.7      0.045   
         1            6.3              0.30         0.34             1.6      0.049   
         2            8.1              0.28         0.40             6.9      0.050   
         3            7.2              0.23         0.32             8.5      0.058   
         4            7.2              0.23         0.32             8.5      0.058   
         5            8.1              0.28         0.40             6.9      0.050   
         6            6.2              0.32         0.16             7.0      0.045   
         7            7.0              0.27         0.36            20.7      0.045   
         8            6.3              0.30         0.34             1.6      0.049   
         9            8.1              0.22         0.43             1.5      0.044   
         
            free sulfur dioxide  total sulfur dioxide  density    pH  sulphates  \textbackslash{}
         0                 45.0                 170.0   1.0010  3.00       0.45   
         1                 14.0                 132.0   0.9940  3.30       0.49   
         2                 30.0                  97.0   0.9951  3.26       0.44   
         3                 47.0                 186.0   0.9956  3.19       0.40   
         4                 47.0                 186.0   0.9956  3.19       0.40   
         5                 30.0                  97.0   0.9951  3.26       0.44   
         6                 30.0                 136.0   0.9949  3.18       0.47   
         7                 45.0                 170.0   1.0010  3.00       0.45   
         8                 14.0                 132.0   0.9940  3.30       0.49   
         9                 28.0                 129.0   0.9938  3.22       0.45   
         
            alcohol  quality  
         0      8.8        6  
         1      9.5        6  
         2     10.1        6  
         3      9.9        6  
         4      9.9        6  
         5     10.1        6  
         6      9.6        6  
         7      8.8        6  
         8      9.5        6  
         9     11.0        6  
\end{Verbatim}
            
    Extracting columns from the data frame:

    \begin{Verbatim}[commandchars=\\\{\}]
{\color{incolor}In [{\color{incolor}38}]:} \PY{c+c1}{\PYZsh{} Extracting the column labeled \PYZsq{}fixed acidity\PYZsq{}}
         \PY{n}{fixed\PYZus{}acidity} \PY{o}{=} \PY{n}{df}\PY{p}{[}\PY{p}{[}\PY{l+s+s1}{\PYZsq{}}\PY{l+s+s1}{fixed acidity}\PY{l+s+s1}{\PYZsq{}}\PY{p}{]}\PY{p}{]}
         \PY{n}{fixed\PYZus{}acidity}\PY{o}{.}\PY{n}{head}\PY{p}{(}\PY{p}{)}
\end{Verbatim}


\begin{Verbatim}[commandchars=\\\{\}]
{\color{outcolor}Out[{\color{outcolor}38}]:}    fixed acidity
         0            7.0
         1            6.3
         2            8.1
         3            7.2
         4            7.2
\end{Verbatim}
            
    \begin{Verbatim}[commandchars=\\\{\}]
{\color{incolor}In [{\color{incolor}39}]:} \PY{c+c1}{\PYZsh{} Extracting the columns \PYZsq{}citric acid\PYZsq{}, \PYZsq{}pH\PYZsq{} and \PYZsq{}quality\PYZsq{}}
         \PY{n}{some\PYZus{}data} \PY{o}{=} \PY{n}{df}\PY{p}{[}\PY{p}{[}\PY{l+s+s1}{\PYZsq{}}\PY{l+s+s1}{citric acid}\PY{l+s+s1}{\PYZsq{}}\PY{p}{,} \PY{l+s+s1}{\PYZsq{}}\PY{l+s+s1}{pH}\PY{l+s+s1}{\PYZsq{}}\PY{p}{,} \PY{l+s+s1}{\PYZsq{}}\PY{l+s+s1}{quality}\PY{l+s+s1}{\PYZsq{}}\PY{p}{]}\PY{p}{]}
         \PY{n}{some\PYZus{}data}\PY{o}{.}\PY{n}{head}\PY{p}{(}\PY{p}{)}
\end{Verbatim}


\begin{Verbatim}[commandchars=\\\{\}]
{\color{outcolor}Out[{\color{outcolor}39}]:}    citric acid    pH  quality
         0         0.36  3.00        6
         1         0.34  3.30        6
         2         0.40  3.26        6
         3         0.32  3.19        6
         4         0.32  3.19        6
\end{Verbatim}
            
    Pandas can show a description of all the data in every column with the
describe() method. This is very useful for getting an overview of the
data in large datasets.

    \begin{Verbatim}[commandchars=\\\{\}]
{\color{incolor}In [{\color{incolor}40}]:} \PY{n}{df}\PY{o}{.}\PY{n}{describe}\PY{p}{(}\PY{p}{)}
\end{Verbatim}


\begin{Verbatim}[commandchars=\\\{\}]
{\color{outcolor}Out[{\color{outcolor}40}]:}        fixed acidity  volatile acidity  citric acid  residual sugar  \textbackslash{}
         count    4898.000000       4898.000000  4898.000000     4898.000000   
         mean        6.854788          0.278241     0.334192        6.391415   
         std         0.843868          0.100795     0.121020        5.072058   
         min         3.800000          0.080000     0.000000        0.600000   
         25\%         6.300000          0.210000     0.270000        1.700000   
         50\%         6.800000          0.260000     0.320000        5.200000   
         75\%         7.300000          0.320000     0.390000        9.900000   
         max        14.200000          1.100000     1.660000       65.800000   
         
                  chlorides  free sulfur dioxide  total sulfur dioxide      density  \textbackslash{}
         count  4898.000000          4898.000000           4898.000000  4898.000000   
         mean      0.045772            35.308085            138.360657     0.994027   
         std       0.021848            17.007137             42.498065     0.002991   
         min       0.009000             2.000000              9.000000     0.987110   
         25\%       0.036000            23.000000            108.000000     0.991723   
         50\%       0.043000            34.000000            134.000000     0.993740   
         75\%       0.050000            46.000000            167.000000     0.996100   
         max       0.346000           289.000000            440.000000     1.038980   
         
                         pH    sulphates      alcohol      quality  
         count  4898.000000  4898.000000  4898.000000  4898.000000  
         mean      3.188267     0.489847    10.514267     5.877909  
         std       0.151001     0.114126     1.230621     0.885639  
         min       2.720000     0.220000     8.000000     3.000000  
         25\%       3.090000     0.410000     9.500000     5.000000  
         50\%       3.180000     0.470000    10.400000     6.000000  
         75\%       3.280000     0.550000    11.400000     6.000000  
         max       3.820000     1.080000    14.200000     9.000000  
\end{Verbatim}
            
    Say you want to find the quality of the wine based upon the amount of
citric acid and the pH value. In pandas you can use
sort\_values({[}'column name'{]}) to sort data by a column:

    \begin{Verbatim}[commandchars=\\\{\}]
{\color{incolor}In [{\color{incolor}41}]:} \PY{c+c1}{\PYZsh{} some\PYZus{}data contains the columns \PYZsq{}citric acid\PYZsq{}, \PYZsq{}pH\PYZsq{} and \PYZsq{}quality\PYZsq{}}
         \PY{n}{quality\PYZus{}sorted} \PY{o}{=} \PY{n}{some\PYZus{}data}\PY{o}{.}\PY{n}{sort\PYZus{}values}\PY{p}{(}\PY{p}{[}\PY{l+s+s1}{\PYZsq{}}\PY{l+s+s1}{quality}\PY{l+s+s1}{\PYZsq{}}\PY{p}{]}\PY{p}{,} \PY{n}{ascending}\PY{o}{=}\PY{k+kc}{False}\PY{p}{)}
         \PY{n}{quality\PYZus{}sorted}\PY{o}{.}\PY{n}{head}\PY{p}{(}\PY{l+m+mi}{10}\PY{p}{)}
\end{Verbatim}


\begin{Verbatim}[commandchars=\\\{\}]
{\color{outcolor}Out[{\color{outcolor}41}]:}       citric acid    pH  quality
         827          0.36  3.28        9
         1605         0.49  3.37        9
         876          0.34  3.28        9
         774          0.45  3.20        9
         820          0.29  3.41        9
         844          0.29  3.30        8
         2775         0.32  3.09        8
         860          0.35  3.22        8
         3188         0.36  3.11        8
         2774         0.32  3.09        8
\end{Verbatim}
            
    \subsubsection{Plotting in pandas}\label{plotting-in-pandas}

Pandas has matplotlib built in, so it is easy to plot a dataframe:

    \begin{Verbatim}[commandchars=\\\{\}]
{\color{incolor}In [{\color{incolor}42}]:} \PY{n}{quality\PYZus{}sorted} \PY{o}{=} \PY{n}{quality\PYZus{}sorted}
         \PY{n}{quality\PYZus{}sorted}\PY{o}{.}\PY{n}{plot}\PY{p}{(}\PY{p}{)}
\end{Verbatim}


\begin{Verbatim}[commandchars=\\\{\}]
{\color{outcolor}Out[{\color{outcolor}42}]:} <matplotlib.axes.\_subplots.AxesSubplot at 0x1e033f23860>
\end{Verbatim}
            
    \begin{center}
    \adjustimage{max size={0.9\linewidth}{0.9\paperheight}}{output_92_1.png}
    \end{center}
    { \hspace*{\fill} \\}
    
    \subsubsection{Creating data frames}\label{creating-data-frames}

You can also create data frames in pandas:

    \begin{Verbatim}[commandchars=\\\{\}]
{\color{incolor}In [{\color{incolor}43}]:} \PY{n}{index} \PY{o}{=} \PY{n}{pd}\PY{o}{.}\PY{n}{date\PYZus{}range}\PY{p}{(}\PY{l+s+s1}{\PYZsq{}}\PY{l+s+s1}{1/1/2000}\PY{l+s+s1}{\PYZsq{}}\PY{p}{,} \PY{n}{periods}\PY{o}{=}\PY{l+m+mi}{40}\PY{p}{)}
         \PY{n}{s} \PY{o}{=} \PY{n}{pd}\PY{o}{.}\PY{n}{Series}\PY{p}{(}\PY{n}{np}\PY{o}{.}\PY{n}{random}\PY{o}{.}\PY{n}{randn}\PY{p}{(}\PY{l+m+mi}{5}\PY{p}{)}\PY{p}{,} \PY{n}{index}\PY{o}{=}\PY{p}{[}\PY{l+s+s1}{\PYZsq{}}\PY{l+s+s1}{a}\PY{l+s+s1}{\PYZsq{}}\PY{p}{,} \PY{l+s+s1}{\PYZsq{}}\PY{l+s+s1}{b}\PY{l+s+s1}{\PYZsq{}}\PY{p}{,} \PY{l+s+s1}{\PYZsq{}}\PY{l+s+s1}{c}\PY{l+s+s1}{\PYZsq{}}\PY{p}{,} \PY{l+s+s1}{\PYZsq{}}\PY{l+s+s1}{d}\PY{l+s+s1}{\PYZsq{}}\PY{p}{,} \PY{l+s+s1}{\PYZsq{}}\PY{l+s+s1}{e}\PY{l+s+s1}{\PYZsq{}}\PY{p}{]}\PY{p}{)}
         
         \PY{n}{df} \PY{o}{=} \PY{n}{pd}\PY{o}{.}\PY{n}{DataFrame}\PY{p}{(}\PY{n}{np}\PY{o}{.}\PY{n}{random}\PY{o}{.}\PY{n}{randn}\PY{p}{(}\PY{l+m+mi}{40}\PY{p}{,} \PY{l+m+mi}{3}\PY{p}{)}\PY{p}{,} \PY{n}{index}\PY{o}{=}\PY{n}{index}\PY{p}{,} \PY{n}{columns}\PY{o}{=}\PY{p}{[}\PY{l+s+s1}{\PYZsq{}}\PY{l+s+s1}{A}\PY{l+s+s1}{\PYZsq{}}\PY{p}{,} \PY{l+s+s1}{\PYZsq{}}\PY{l+s+s1}{B}\PY{l+s+s1}{\PYZsq{}}\PY{p}{,} \PY{l+s+s1}{\PYZsq{}}\PY{l+s+s1}{C}\PY{l+s+s1}{\PYZsq{}}\PY{p}{]}\PY{p}{)}
         \PY{n}{display}\PY{p}{(}\PY{n}{df}\PY{o}{.}\PY{n}{head}\PY{p}{(}\PY{p}{)}\PY{p}{)}
         \PY{n}{df}\PY{o}{.}\PY{n}{plot}\PY{p}{(}\PY{p}{)}
\end{Verbatim}


    
    \begin{verbatim}
                   A         B         C
2000-01-01  0.925911  0.826709  1.265237
2000-01-02  1.238019 -1.119733 -0.936616
2000-01-03 -0.898996 -1.347025 -0.604194
2000-01-04  0.717334 -0.597611 -1.444401
2000-01-05  1.718913 -2.318578 -0.250939
    \end{verbatim}

    
\begin{Verbatim}[commandchars=\\\{\}]
{\color{outcolor}Out[{\color{outcolor}43}]:} <matplotlib.axes.\_subplots.AxesSubplot at 0x1e033f632b0>
\end{Verbatim}
            
    \begin{center}
    \adjustimage{max size={0.9\linewidth}{0.9\paperheight}}{output_94_2.png}
    \end{center}
    { \hspace*{\fill} \\}
    

    % Add a bibliography block to the postdoc
    
    
    
    \end{document}
